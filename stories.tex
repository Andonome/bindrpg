\chapter{Stories}\label{stories}\index{Stories}

\begin{multicols}{2}

\noindent
Players `write' most of their backstory during play rather than before it.  \Glspl{pc} can start off as blank slates with no history, but the history comes out of the woodwork soon after as players can spend 5 Story Points to bring their history into the current adventure.  Let's look at an example:

\paragraph{Session 1} has the characters running from the local law.  Jane's player spends 1 Story Point and declares \textit{``Luckily, Jane has connections with the local thieves' guild, so she nips into an alley in the slums where the party can lay low for a while''}.

Soon after, the characters need to sell the diamond they've stolen.
Eric's player knows there are dwarves all around town so he tells the group \textit{``I'll see about help from the local dwarves.
I learnt their language five years ago when I had to hide from the law for a while underground, and met a few of them that sell goods without much question''}.

\paragraph{Session 2} finds the characters lost in deep, long caverns in the bearded mountains, wounded and low on supplies.  Jane's player spends 4 Story Points to declare she knows of a gnomish illusionist who frequents these deep caverns, looking for magical ingredients.

\textit{``How?''}, asks a rather suspicious \glsentryshort{gm}.

\textit{``Well, he used to work with the thieves' guild when he was younger, helping us steal with his illusion magic; sometimes he would give us a magical item which would cast an illusion of something we wanted to steal, so nobody would notice it was missing for a while.
The guild kind of fell apart after he left, which is why it's nothing but slumlords and cutthroats now.''}

En route to a dwarven stronghold with Jane's illusionist, the band are assaulted by a small army of goblins wielding strange magics.
A dwarvish outpost is nearby, so the group run and bang on the great iron gate.  Eric's player spends 2 more Story Points, saying \textit{``This is the place I stayed -- they all know me here.  They should let us in, help us with some supplies, maybe even get me a new sword''}.

At this juncture we know a fair amount about Eric and Jane, where they come from, and who they are, while Sindon the elf continues to be a mystery.

\paragraph{Session 10} comes after some downtime.  The group are lost in a mysterious forest, now teeming with the undead.  Their arrows and rations have run out, the trail leading to the necromancer has gone cold, and they don't think they have the strength to defeat him even if they could find him.

Sindon's player decides to spend his 7 saved up Story Points.

\textit{``You don't happen to know any elves in the forest, do you, Sindon?''}, asks Jane.

\textit{``Yes''}, says Sindon.  \textit{``It's been twenty years, so I suppose it's time to go and see my father.  You can come out now''}.

In the distance, six elves come out of hiding and walk towards the characters.
Sindon's father turns out to be a local warlord.
The party receives whatever supplies they wish, a scout has recently found the necromancer's lair, and four elven warriors agree to accompany them to fight the necromancer.

\end{multicols}

\section{Story Points Rules}

\begin{multicols}{2}

Players begin each with 5 Story Points and spend them at any point during the game. The encounters must take place in a rational manner -- players might find the perfect sellsword in a town, but if they're in a dungeon, fighting a hall of ghouls, there's little reason for a random sellsword to be present and looking for a job -- this is not an ability to magically summon useful tradesmen with a flash of smoke and plot. As a result almost all stories will have to be told in populated areas such as towns and villages.

The \gls{gm} is, of course, free to veto any Story suggestions without explanation in order to maintain the integrity of the plot or stop cumbersome play issues.

All stories should be noted down on the back of the character sheet, including any stats from companions, just in case they enter during a later adventure.

\subsection{Combining Stories}

Whether telling one story each adventure or letting everyone know all about your character's backstory all at once, players are encouraged to think about weaving their stories together. You may have told us that you learnt gnomish when staying with the gnomes. Now that you need a blacksmith in this village, why not specify that he's a gnome whom you once knew? \ And if you need a sellsword to join your group later, how about specifying that you once fought with him to defend the gnomes?

Alternatively, if you are taking out all your stories at once, you might want to declare that you know a mage who lives in a place you can access through a nearby secret portal. You instantly adopt a safe space and a helpful magical ally, then start expounding upon the days when the alchemist was proudly telling you about his impregnable home.

\subsection{Downtime}

`Downtime' is when the current stories come to a close and the \glspl{pc} take a rest.
It can be weeks, years, or even decades.
During a particularly long Downtime the \gls{gm} may grant the players an additional Story Point, or even multiple Story Points for a downtime of many years.

Some characters may save up their Story Points at this juncture just to buy something expensive later.  Alternatively, characters can use those points to explain what they were doing during the \gls{downtime}.  Perhaps the group earn fabulous wealth and split up for some years, then upon returning one of them has learned dwarvish, while another joined the military and gained friends willing to help out on some new quest.

The party can declare \gls{downtime} at any point, although the \gls{gm} is free to interrupt that \gls{downtime} with events.
Likewise, the \gls{gm} can declare a \gls{downtime} at any point, but the players can interrupt this with personal missions.

\end{multicols}

\section{Sample Stories}

\begin{multicols}{2}

\noindent
The following is a suggested list of Stories the players can tell and their costs. The players are strongly encouraged to suggest more to the \gls{gm} who will either veto them or give them an appropriate cost.

\story{1}{Perhaps we can make a detour}
You know of a sacred location nearby, perhaps a church, or a shrine or just a sacred cavern where the land is teeming with magic.
In this sacred area, anyone stepping into it receives 1 \gls{mp} per \gls{round}.
If the spot has a guardian then they are friendly to you.
The place will not necessarily help you hide or defend yourself unless you are also spending Story Points to make it a place to rest.
\footnote{Those following the Code of Experience gain no \glsentrytext{xp} for finding this location.}

\label{oldnpc}
\story{1}{Oh! Don't I know him}
You recognise a friendly character from some previous Story you have told. The \gls{gm} will explain why they are in town but you are free to offer suggestions. Said characters won't necessarily be as useful as they would be if they were brought into the adventure for the first time with Story points and may only help for a scene, but they should be somehow useful. This may include a trader who was previously known to have valuable information about some situation, or a mage the characters had previously met who could cast a useful spell or two.

This \gls{npc} will probably have gained some \glspl{xp} over this time.
The \gls{npc}'s \glspl{xp} is still equal to half the total \glspl{xp} of whichever party member has the highest \glspl{xp} total.\footnote{Although this cannot make the \glspl{xp} lower than it was.}
Any additional \glspl{xp} must be spent immediately (spare XP is discarded), with an explanation about what happened to acquire these new Traits.

\story{1}{I think I heard something about this}
When the \gls{gm} asks you to make a check to gain knowledge, you can spend a Story Point and mention how you know this one particular fact about this topic. You gain a +6 bonus to a single knowledge check. This does not count again for the same domain of expertise -- it is only a bonus to knowing one, single fact about the subject.

A failed roll indicates that while you have a lot of history intertwined with this problem, you are still wrong.

\story{1}{My uncle taught me something about this}
You have a surprising Skill or Knack which will comes in useful. As you tell this story, you can buy a Skill level so long as you have the requisite \gls{xp}. This cannot be a Skill which you have clearly lacked in the past, e.g. if your character has so far been illiterate then you cannot suddenly learn a level of Academics. However, if you have never wanted for Craft ability then you could declare that you have always known how to forge iron, or that you have a Seafaring Skill.

\story{1}{Fun fact about the elvish first person plural}
You have spent a significant amount of time in another culture. You know their language and enough of their background to transfer over basic Skill knowledge. If you have the Performance Skill and are familiar with elvish culture then you also know some Elvish songs. If you are familiar with gnoll culture and have the Empathy Skill then you know a range of details about gnoll etiquette and lineage.

\story{1}{It'll be just like the old days, remember that time}
At the point a new character joins the group you can select one other player and have a shared background with them (or with another, if your character is new). You describe how you previously met and possibly adventured together. From then on, you can split the cost of stories, so if the group wants to find a safe space to rest then instead of one character spending 2 Story points you could each spend 1. Each of you can use characters from the other's background, because all your Stories have the option of being shared stories. If you are both of noble heritage, any money you get must be divided between you. If you are both friends with a skilled armourer, they will only be able to repair one piece of armour at a time.\footnote{This Story is transitive and symmetrical, so if player A shares a background with player B and player B shares a background with player C then player C also shares a background with player A.}

\story{2}{I know a guy who'd be perfect}
You know someone in town who has just the skills you are all looking for.
They might be a farmer, willing to put you and the group up for the night, or someone who knows all the local rumours.

The player can make this character themselves, just like a normal character.
For the sake of brevity, consider using the rules for player chosen characters (page \pageref{playerchosen}).
The \gls{npc}'s starting \glspl{xp} is half the current \glspl{xp} total of whichever party member has the highest total \glspl{xp}, so if the highest \glspl{xp} total in the party is 83, that's 42 \glspl{xp} to make this character.

This \gls{npc} refuses to join the party on any martial escapades unless this is paid for with a grand story (see below).

This is a particularly important story, as these form the secondary characters which players can use if their first characters die.

\story{2}{I know a place we can rest}
You know of a secluded and secret location where you will be safe. Perhaps there is a safe spot in a tavern you know -- a secret room in the basement, or maybe just an abandoned and deep cavern in the hills that nobody knows about.

If your safe space is ever invaded due to events outside your control, you receive both Story points back if it is within the same session or 1 Story point back if it during a later session where the same place is used again.

\story{3}{Ah! This is near the spot we buried the treasure}
You have access to large funds now that you have returned to this area.
Perhaps you and companions, once buried treasure close by.
Perhaps a local bank simply has your money, or a rich man owes it to you.
The total amount obtained is $2D6 \times 10$ gold pieces.\footnote{Those following the Code of Acquisition gain no \gls{xp} for gaining gold through Story Points.}

\label{tim}
\story{4}{There is a man whom they call}
Your miraculous ally is a mage, or priest or some other \gls{miracleworker}.
They will not enter combat with you but will agree to employ whatever magics you wish.
They have the standard \glspl{npc} attributes from Story Points.
Additionally, their minimum Intelligence Bonus is +2, and they have one magic sphere at 4th level, another at 3rd level, and another at the 2nd level.
They walk a single path of magic.

\story{5}{Do you know who I am!?  Because I happen to be}
You are the child of minor nobility -- perhaps a knight errant or son of a Town Master.
You can collect $2D6 \times 5$ gold pieces from your homeland.
If, on the other hand, the money is yours then you can start with it by taking this Story when you begin play but cannot ever double your money by asking parents for more.
You have access to a minor keep -- either your own or a parent's -- and can demand the services of any skilled tradesman in the land except for magical talents.
You do not have special military access but can buy their swords for the usual rate.

\story{7}{My father will give us a royal welcome when we get to}
You are revealed to be the child of royalty or some other type of nobility, and you are returning to your kingdom.
You can request almost anything from the royal family, within reason, including $3D6 \times 10$ gold pieces.
While with your family, you can use up to 6 story points each adventure. With these you can purchase men at arms, demand the help of tradesmen or any other story except for learning a language.

Since this Story costs 7 Story points, no player should expect to use it until there has been some downtime -- if the character starts out claiming to be a prince then it will be a long time before the story recognises this claim.

\subsection{Grand Stories}

Grander stories can be constructed by modifying existing stories.
Standard tales can be given some bonuses at an additional cost.

If the \gls{pc} knows someone around the village, multiple \glspl{npc} could be introduced -- perhaps because this is the \gls{pc}'s home town, or the \gls{pc} saved the village some time.

\story{1}{\ldots and he's willing to fight with us}
The character from another story is willing to fight with the party for one mission.
The party must make their way directly to the mission, and those willing to fight will leave after a single combat encounter, whether or not that was the chosen encounter.

\story{2}{Fancy seeing you here}
You can add two to the cost of any other Story and tell it at an inappropriate juncture. Your characters might be locked in a dungeon and happen upon a weapon smith in the next cell, with his confiscated weapons lying in a nearby pile outside his cell. They might find a place they can rest in secret inside the terrifying dwarvish city turned into an undead haunting ground. Or perhaps while on the run from bandits they find a helpful soldier hoping to be hired.

\story{2}{Actually, there were a few more}
Increase the cost of any story you tell by 2, and raise the number of people or places to the number of story points you're spending plus your Charisma Bonus.
If telling the story of a safe space nearby, you might know about a few.
If you know a sellsword interested in adventure, you might increase this to four people, so you now know four people ready to join you on a mission.

Each location or person must be specified immediately.
If you have money, each source of money must be in a different location.
If you know where multiple mana lakes are, you must provide a good reason so many rare places are so close together.


\end{multicols}

\iftoggle{verbose}{
\section{The Call to Adventure}

\begin{multicols}{2}

\noindent
Just because characters can begin as blank slates doesn't mean the story can.  A basic premise can help tie those backstories together.

If the \gls{gm} has no definitive plans laid out for the campaign, players should suggest a good starting point.

\subsection{The Night Guard}

The world of Fenestra doesn't have many wars or diseases, but it never becomes overpopulated.  The reason is simple: monsters.  There are giant arachnids in the forests, basilisks which belch poison and steal cattle, and the occasional dragon.  If someone can't find a useful way to employ themselves, the Night Guard awaits.

The majority of the Night Guard have boring, thankless jobs such as guarding cattle, patrolling for goblins, and occasionally clearing an area where suspected monsters guard territory.
A few go onto more dangerous jobs such as hunting ogres, tracking down criminal gangs, or espionage.

Some rare few have been known to strike deals with dragons to leave an area, or assassinate rogue alchemists who are powerful enough to keep themselves free from the reaches of any magical guild.

In general, the more dangerous and skilful the job, the higher the pay, so most of the Night Guard try not to do too well at their job.
They train in archery well, take a pay cut in return for having more members in their group, and make sure nobody volunteers them for anything interesting.
Of course, a lot of the jobs one takes depends more upon a captain of the Guard than the grunts.

\subsection{The Illusionists of the Bearded Mountains}

The characters are all gnomes, defending against encroaching goblins who have come through a portal, while the elders constantly argue that if only \emph{somehow} someone could get down there and destroy the portal, everyone would be safe.
But that ``somehow'' never comes, and the monsters are coming up faster and faster.
Rumours abound of distant elves who might help, but those elves have their own problems.
Meanwhile, the daily lives of the warrior-alchemists consists in setting and resetting various traps made of pitfalls, illusions and rope.  Each day the gnomes have to retreat farther from the deeps and closer to the Sun.

Once each member of the group has expended three Story Points, an opening comes to travel to the nearby elves, and beg for an army to save your homeland.

\subsection{The College of Alchemy}

The characters are all alchemists in the service of the Alchemist's Guild in Eastlake.
The first part of the campaign involves high-school rivalries against other Clans in the guild such as stealing their homework, or vying for romantic attention.
Soon after, the characters begin proper guild missions, venturing out into the strange areas of the world where normal people will not tread.

At the campaign's start, characters get only 1 Story Point each.
Each year's Summer holidays grants an additional Story Point, so each character will have the full 5 Story Points at the end of the four year course.

\subsection{The Game Changer}

The party stumble across a game-changing magical item, capable of raining fire down on an entire battlefield.
Its use to them will be limited, because the magic in the item takes a long time to use, but the effects for the kingdom's war will be huge.

While the war is far away, both side know the party have the item, and want to take it from them.
Individual lords attempt to grab the item in order to take it to a king themselves, or even to start a rebellion against the king while the war is on.

As the party moves the item to the battlefields, they hear more and more rumours about the activities of the kings on each side.
By the time they arrive, they will able to pick a side.

\end{multicols}

}{}
