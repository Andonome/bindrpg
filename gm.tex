\chapter{Games Masters}

\section{Basic Prep \& Play}

\begin{multicols}{2}

\noindent
The basic tools of the Games Master must begin with with the obvious -- $4D6$ per player with multiple $D6$ colours so players can differentiate their Damage dice from their Action dice.
Next, of course, one character sheet and pencil for each of the players.
Since this can be a lethal game, especially for new players, consider adding a few `just in case' character sheets.

\subsection{Coins}

To helps players understand the tactical elements of the game, consider setting a central initiative track on the table, with the number 1-18.
Have everyone place a token, model, coin, or whatever, on their own Initiative number as soon as combat starts so that they can see the Initiative count moving slowly towards them.

As a \gls{gm}, it's always good to have at least 3 different types of coins.
Let's say you're orchestrating a battle with a hobgoblin leader, some hobgoblin troops and a goblin spellcaster.
Assign each one a coin and make a little mnemonic -- the spider has dark skin so it gets the little copper penny.
The hobgoblins get the silver coin to represent their use of weapons, and the largest coin goes to the hobgoblin leader.
Don't worry about the players' Initiative -- they'll keep track of their own characters as you shout out where on the Initiative tree you are.

Coins should also be used when assigning the Combat Skill.
The character sheets contain a large space in the middle where players can add bonuses to their Combat Factors rather than attempting to remember where everything was placed.

Coins can even be used to keep track of \gls{fp} and Fatigue Points as they change so often.
It'll help cut down on wear to the character sheet.

\end{multicols}

\section{Encounters}\index{Encounters}

\begin{multicols}{2}

\subsection{Random Encounters}\label{encounters}

Whether you're in the middle of an adventure or the \glspl{pc} are just randomly wandering the world without any respect for local laws or plot, a random encounter can always add a sense of danger to a non-urban area.

Each time the players pass through a region, roll $3D6$ on the encounter table and create an encounter from the result.
You can make a unique encounter table for each region in your campaign to individuate them.
As an example, have a look at Redfall's forests:

The forest can be a dangerous place, but not nearly as dangerous as the marshes.
The entire Redfall area is infested with ghouls, but they get much more common once one passes beyond the forest's edge and into the marches.

Some encounters presented are fairly benign.
Wolves may try to steal the party's food, but they're not dangerous, and human traders simply provide an opportunity to gain news, and travel with a little more safety.
Despite the different tables, the overlap provides some cohesion to the area.

\begin{encounters}{Redfall}

Marshes & Forest & Result \\\hline
	\li & Elven fortress. \\
	\li & $2D6-1$ elven hunters. \\
	\li & $1D6+5$ Hobgoblins. \\
	\li \lii $3D6-2$ Ghouls. \\
	\li \lii $3D6-2$ Goblins. \\
	\li \lii $1D3$ Griffins. \\
	\li \lii $2D6$ Bandits. \\
	& \lii Bear. \\
	& \lii $2D6$ Wolves. \\
	& \lii $2D6-1$ Human traders. \\

\end{encounters}

If you reach a result which is not listed, there is no encounter.
If you roll trips -- three of the same number -- roll again, and if you get another encounter, combine the two.
If you get a griffin and a bandit, perhaps the players stumble upon bandits in the woods, attempting to pilfer griffin eggs for a patron.
If you roll wolves and a chitincrawler, perhaps the players hear persistent wolf-cries in the distance as a chitincrawler has caught some wolves in its web while the others watch and bark helplessly.

You may want to set up your random encounter before the start of the session, allowing you to review monsters' stats and perhaps tie the encounters together, or integrate them with active characters from players' Story Points, or recent events in the campaign.

If you have a campaign book such as \textit{Adventures in Fenestra}, you'll find stats for creatures, suggested encounters, and random encounter tables for the different areas.

\subsection{Side Quests}\label{sidequests}

Another way to add impromptu elements into your game is Side Quests.
These are short encounters which slowly feed elements into the background of your game.
They're good for foreshadowing without too much planning, and good for adding things to the path of players who simply want to run around in a sandbox, without the constraint of a full-on plot-arc.

Let's look at an example from a village area:

\begin{exampletext}

	Villagers have been cutting down trees near a spot sacred to the elves.
	Negotiations have failed, and now the elves intend to drive the humans out like vermin by burning down their houses.

\end{exampletext}

\paragraph{Encounters:}

\begin{list}{\Square}{}

\item[\CheckedBox]{Villagers are burning a witch at the stake and will grab any known magic user or elf in the party.}

\item{The party notice a group of elves sneaking up to a village. If they don't stop them, the elves attempt to set fire to various houses.}

\item{Watchmen arrive in the villages, with orders to kill all magic users and elves on sight. Repeat.}

\end{list}

When springing this Side Quest on your players, you start with part 1.
The second time the players encounter this Side Quest, do part 2, and so on, until the encounters have finished.
Some (such as this) finish on a loop, so the players can repeatedly encounter Watchmen in the village who will not take kindly to known magic users.

Notice that none of the encounters require the party to do anything.
If they don't want to engage in the plot, they can sit back and watch, except insofar as the villagers have a problem with them.

One more example:

\begin{exampletext}

A priest is using his ability to divine the future to capture criminals \emph{before} they commit crimes.

\end{exampletext}

\begin{list}{\Square}{}

\item[\CheckedBox]{A local priest offers to tell the party their fortunes.  Combine this with the next encounter, then move it to Town.}

\item{(Town) The characters pass by men in stocks who keep shouting that they are all innocent, and were suddenly taken away by various guards after the local priest fingered them for a crime.  Move this encounter back to the villages.}

\item{A dozen guards are tracking the characters. Repeat.}

\end{list}

The characters are now wanted by the guards who wander the villages, hunting for would-be criminals.

\subsubsection{Summary}

When the players enter the villages, they encounter a Side Quest -- perhaps the next part of the priest's story, perhaps the next part with the elves.

The first encounter combines with the next Side Quest (whatever it happens to be).
This helps Side Quests integrate, and adds a little more action to would-be slow scenes.
The second encounter moves to the pile of town Side Quests, so it can only be encountered there.

Side Quests should never require characters going to a specific location, since they are something which happen \emph{to} the party, but Side Quests can still reference a details area, such as the local priest's church, or the sacred lake which the elves guard.

If you want to run Side Quests as a secondary part of your game, you can just run them any time the group doesn't get a random encounter.

If you want them to be the primary mover in your campaign, you can run a Side Quest every time the group enters a new area.
You can also make one plot line the \emph{primary} quest by making it longer than the others.

However you run them, players should each receive 5 \gls{xp} for completing a Side Quest for each part it contained.  A 2 part Side Quest grants 10 \gls{xp}, while a 4 part Side Quest grants 20 \gls{xp}.

\subsubsection{Anatomy of Side Quests}

Side Quests often begin with an example to introduce the players to the scene.
This example won't work for every group in just any situation, but provides a starting point to picture how things might play out.

\begin{boxtext}

	As you sit down to write your first Side Quest, you are assaulted by a blank white page!

\end{boxtext}

After that, you'll find details such as the \glspl{npc}, with their stats and motives.

After the Side Quests have finished, you'll find details of any locations relevant to the Side Quests.

\subsubsection{Preparation}

Rolling up Encounters and Side Quests beforehand can really get a game rolling, and you'll have more opportunity to integrate those encounters together.
You'll find space on your GM sheet (back of the book) to write down a couple of Encounters and Side Quests per area.

Once a Side Quest becomes available, tick the box next to it in the miniature table of contents (the first one is ticked by default).
Once you have completed a part, mark it with an `X' and dish out the \gls{xp}.

\end{multicols}

\section{The Undead}

\begin{multicols}{2}

\noindent
Undead creatures have certain properties in common. Firstly they imperceptibly feed from the souls of the living. This is not performed with the mouth by merely by being close to dying things and absorbing them before they can wander to the next realm. Undead eyes generally do not work, instead they `see' the souls of people shining outward. Inanimate objects such as books, or even fellow undead, are not so clearly seen; the undead can avoid bumping into these objects but have great trouble reading anything or working fine machinery. However, they can operate in complete darkness and even fight without penalty, using the light of living people's souls to see them. They can also see living beings from a great distance due to the soul-light they emit.

Undead also feel no pain and suffer little from scrapes and bruises. As a result, they automatically have a \gls{dr} of 2 which is cumulative with armour. This counts as Complete armour, but not Perfect -- shots through their eyes or attacks which sever muscles still debilitate them.

The undead do not tire -- they take no Fatigue Points. They can walk or dig or fight endlessly, without complaint.
They enjoy feeding on souls, but it is not required for them to continue moving.
Each has an Aggression score of +2.

When the undead are newly created, they are clumsy, as they are not used to their own bodies, and suffer a -2 penalty to Dexterity.
Shortly afterwards, rigour mortis sets in, and then decay.
Any undead more than a few hours old gain a -2 penalty to their Speed Bonus, but lose the Dexterity penalty.

\end{multicols}

\section{\glsentrytext{gm} Suggestions}

\begin{multicols}{2}

\subsection{Fast Initiative \& Good Pacing}

You can give a good pace to combat by hollaring the Initiative count.

\begin{quote}

``Twelve! The gnolls ready their weapons''

``Eleven, ten! They move forward, bearing their yellowed teeth.''

``Nine! Snarls abound as they speed up to a rush.''

\end{quote}

Nothing has actually happened by this point, but it sets the scene nicely.

\begin{quote}

``Nine'', one of the players shout.  ``I'm going at nine.  I move to protect Max.''

``Two gnolls go for you, another two go for Amelia.  Roll to defend at TN 11.''

\end{quote}

The initiative continues down quickly at all times, and the count always provides a sense of urgency.
If players don't notice it's their turn when you're shouting, that's 1 Initiative point lost.
Do it once, and they'll never make the same mistake again.

\subsection{Damage, Death \& Dismemberment}

Losing HP is a massive, screaming deal in BIND.
It's easy to take habits over from other games where losing one's liver is all part of a normal Tuesday afternoon but here \glspl{pc} should lose \glspl{fp}, then attempt to flee and only in the most dire situations should they start to bleed.
Damage which doesn't hit home can be brushed over with a brief note about `avoiding the swing' but if anyone loses a single Hit Point the \gls{gm} should grind the description and combat to a halt to emphasise exactly how eyeball poppingly, knee-cap shatteringly painful and side-splittingly debilitating a knife can be.
Take your time.
Make the words secrete congealed blood.
If the \glspl{pc} start to lose HP and don't realise how serious this situation is they might perish where they otherwise would have run away to fight another day.

If a \gls{pc} dies, the player should be slotted into the adventure at the next available opportunity with a new character.
If there is no plausible way to insert another character any time soon, consider providing an \gls{npc} for them to play - it doesn't need to be one with amazing Traits, just someone who can speak and interact with the world.
All unspent \gls{xp} from the old character should be given to the new one, allowing the player to boost the N\gls{pc} or save up more for when they finally make their own character.

If the \gls{pc} is wounded to the point of being useless then that player is not going to have a lot of fun with the character.
If possible, the player should be given a new character for that one adventure, but all \gls{xp} gained can be kept and given to the old character at the end of the adventure.
To put it another way, players, rather than characters, hold \gls{xp} values which can then be placed on any character.

\subsection{Rollplay Before Roleplay}

It's hard to play `the social character'.
You put all your \gls{xp} into a high Charisma score because you want to build alliances and understand people, then the \gls{gm} asks you to roleplay the encounter and all that comes out is your natural stutter.

It's also hard playing a non-social character.
You have been lumped with a character with a Charisma Penalty of -4 and by all the gods you intend to roleplay it, so it's time to ask the town master which lady he stole his robe from and then wipe your mouth with the tablecloth.
But the other players are not impressed; all they can see is someone intentionally ruining the encounter rather than the fun-loving, amazing improviser that you are.

Consider the following solution: tell the players that if they wish to speak, they must roll Charisma plus Empathy or Wits plus Whatever, then set the \gls{tn} for the encounter.
Getting information from the drunken patron of a temple of Alass\"{e} might be \gls{tn} 4 while getting a noble to stop and give everyone a hand might be \gls{tn} 10.
The player should not declare the result but make a mental note of the roll's Margin.
If the Margin is high, they should confidently roleplay someone saying just what the situation appears to demand.
On the other hand, if the roll was not only a failure but had a high Failure Margin, they should attempt to roleplay the worst kinds of insults -- perhaps because the character is genuinely mean-spirited, perhaps because they are making persistent, accidental faux-pas.

This method of players rolling before roleplaying to indicate their roll gives value to the social characters' Traits and legitimacy to the antics of more socially clumsy players saying all the wrong things.
The roll of the dice also acts as a way of saying `I am about to speak', so people can pace conversation without interruption.

\subsection{\glsentrytext{npc} Fights}

\begin{exampletext}

	``The goblin platoon start throwing more spears, but then from the side, the garrison of guards burst into the cavern's entrance to join you.''

\end{exampletext}

Add a few too many \glspl{npc} to a fight and you can end up either being a stumped \gls{gm} or having players wait for you to roll an awful lot of dice on your own.

If you need a quick approximation of a massive battle, just have each \gls{npc} deal its own \gls{xp} value in Damage each round (ignoring \gls{dr}).
A guard worth 10 \gls{xp} who fights with the characters deals 10 Damage, which could mean killing a single creature with 10 Damage, or could mean finishing off 2 creatures the characters have already wounded, by dealing each one 5 Damage.

\begin{exampletext}

	The \gls{gm} thinks for a moment.
	That's 30 goblins and 12 guards.
	The twelve guards deal 10 Damage each, killing 10 goblins, then the 20 remaining goblins deal 20 Damage, killing 2 guards and wounding another.

\end{exampletext}

If two \glspl{npc} fight, whichever individual has the highest single \gls{xp} deals Damage first.
So if ten soldiers worth 10\gls{xp} each fight a basilisk worth 24\gls{xp}, the basilisk would deal 24 Damage, killing 3 soldiers.
On the next turn, the 7 remaining soldiers would deal 70 Damage, killing the basilisk.

\begin{exampletext}

	``The guards spill in, massacring the goblin horde.
	You see some surrounded, and spears driven into them, but the rest keep fighting.''

\end{exampletext}

Obviously, this system is not going to represent anything with much accuracy, but it's better than halting a game so you can roll dice for twenty minutes alone.

\end{multicols}

