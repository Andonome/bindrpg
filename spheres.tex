\chapter{Spheres}

\begin{multicols}{2}

\noindent
A novice miracle worker begins by selecting one of the five paths of magic.
Each path grants access to spheres of magic, i.e. collections of spells.%
\footnote{See chapter \ref{magic_paths} for the paths of magic.}

Each level of a sphere typically grants access to a few different spells.
For example, the first level of the Aldaron sphere allows the caster to affect local weather conditions, enchant animals, and summon light.
Divine casters will think of this as a gift from their deity, while blood casters think of these effects as a natural extension of their own will.
However, the basic effects are the same.

\subsubsection{The Spheres of Magic}

\paragraph{Aldaron} allows one to enchant animals then later to harness control of the local weather conditions.

\paragraph{Conjuration} changes things from one form to another, and eventually can summon items out of the air.

\paragraph{Enchantment} allows casters to calm people or panic people. How to confuse and impress them.

\paragraph{Fate} is divine magic and allows the caster to ask a question of the gods, then later to heal companions' \glsentrylongpl{fp}.

\paragraph{Force} magic is a very versatile sphere, allowing the mage to protect themself, fight with levitating weapons or just levitate any object or person.

\paragraph{Illusion} allows the caster to summon apparitions of anything. The caster might hide a door by making an illusion of a wall over it, or create the image of a sleeping bear to frighten people. More skilled illusionists can disguise themselves as other people or creatures.

\paragraph{Invocation} is the magic of fire, lightning and destruction. It begins with bolts of lightning and later allows the caster to incinerate large swathes of enemies with great balls of fire.

\paragraph{Necromancy} first deals with making the caster close to death so they can feel no pain and interact safely with the risen dead.
Later the necromancer learns to summon simple spirits into the bodies of the dead to make them rise as an army.

\paragraph{Polymorph} allows the caster to transform into other races, and then into entirely different species.
Exactly which type of animal a caster can transform into depends upon their body type.
Lithe characters will find it easier to turn into a bird, while stronger people will find stronger animals, such as bears or warthogs, easier.

\end{multicols}

\resumecontents[magic]

\sphere{Aldaron}

\begin{multicols}{2}

\noindent
The elves are intimately familiar with this sphere, and usually refer to it as a simple skill, like painting or any other trade. They call it simply `the knowledge of trees', though it deals with much more than wood -- animals can be turned into friends and companions, the weather can be controlled and at the ultimate level the forest itself can be called to uproot and give aid to the mage.

\spelllevel

\label{forestsong}
\spell{Forest Song}{Continuous}{Beast Ken}{Enchant animals as per levels 1-3 of Enchantment}
\noindent
Novices of Aldaron can befriend any beast, make them confused, send them to sleep or send them into a blind panic.
Passive mammals such as sheep are easy to target while aggressive or strange creatures can be very difficult to get to grips with.

The \gls{tn} for this spell is 7 plus the target beast's Wits + Aggression Skill (the Skill which replaces Combat for beasts). The caster rolls their Intelligence + Beast Ken. For example, a creature with Wits +1 and Aggression +2 would be at \gls{tn} 10 to affect.

Mages can use this magic to make animals easier to train, although most animals are not particularly useful -- they cannot tell the mage important information or understand simple commands.

Forest Song works on all creatures without an Intelligence score.
Umber hulks, bears, birds, et c. -- all can be affected with the language of the forest.
However, mammals are the easiest to work with.
The \gls{gm} should add to the \gls{tn} to affect birds, insects and other non-mammalian creatures.

Forest Song replicates the first three levels of the Enchantment Sphere but the targets are beasts rather than people, the caster always uses the Beast Ken Skill.

\enhancement{1}{Binding}{Replicate all 5 levels of Enchantment}

With an additional level added, the spell can replicate all five levels of the Enchantment sphere, but retains the exception that the only Skill used is Beast Ken.
The animals targeted by this spell do not become any smarter, unless the enhancement \textit{Sentient} is used with the spell.

\spell{Light}{Continuous}{Survival}{Blind enemies and light the way}\\
\label{light}

\iftoggle{verbose}{

	\toppic{Roch_Hercka/flashing_light}{\label{roch:light}}
}{}

\noindent
The mage casts a dim light, about the strength of a torch, which floats around a single point (but never very steadily).
This light can blind an opponents in the darkness by casting it directly in their face.
Anyone having the werelight flare up in their face becomes blinded for a number of rounds equal to the spell's level minus the target's Wits Bonus.
The average human, having a Wits Bonus of -1, would be blinded for 1 round.
The blindness can be automatically avoided by anyone who was Keeping Edgy (see page \pageref{edgy}), as quickly shielding one's eyes averts any damage.

Undead are terrified of this light.
Those affected by the spell make a Wits + Aggression roll, \gls{tn} 7 plus the caster's Intelligence + Survival.

\spell{Plantform}{Continuous}{Survival}{Change a plant's natural adult form}\\
Young plants have a natural destinty.
With this spell, a plant's destined form can be changed.
The caster needs to hold the spell until the plant has fully formed, which can stunt the caster's mana for a year or more.
The affected plant cannot be larger than a man, unless enhancements increase the area of effect.

The caster has various options for how the spell grows the plants:

\paragraph{Edible} plants produce a number of meals equal to the spell's level plus the caster's Intelligence Bonus.
\textit{Wide} spells produce the same amount of food times the spell's level, plus the caster's Wits Bonus.

\paragraph{Poisonous} plants taste the same as the edible plants, but inflict a number of Fatigue Points when ingested equal to the spell's level plus the caster's Wits times 2.\footnote{$(L + Wts)\times 2$}

\paragraph{Wildform} plants are just plants with any shape the caster desires.
They might grow into the form of a chair, or even a house if the spell is large enough.
Anything is plausible if a plant could be carved into the right space.

\spell{Freezing Touch}{Continuous}{Survival}{Turn water to ice or freeze someone's body}\\
The mage can freeze solid any body of water, or even damage people by cooling their body.

If cast on a person, they take \arabic{spelllevel} Fatigue Points plus the caster's Intelligence Bonus.\footnote{The elvish natural immunity to cold does nothing to prevent this damage.}
Exactly how effective this is depends a lot on how tired the target already is.

Bodies of water freeze over the moment the spell is finished.
Such ice has an effective Strength Bonus of \arabic{spelllevel} plus the caster's Intelligence Bonus, and covers up to \arabic{spelllevel} squares plus the caster's Wits Bonus.
The spell's Strength Bonus can test if the ice can trap people who are in the water, or if it can support people's weight (it holds a maximum \gls{weightrating} of its own Strength +4).

Creatures only frozen up to their waist or ankles can gain a bonus to break out of the ice, and a further bonus if the spell is cast slowly.
If the caster can extend the range, then the spell can travel any distance, although longer distances can make the spell rather a long-shot, with each area traversed raising the \gls{tn} by 3.

\spell{Wind Blast}{Instant}{Survival}{Push enemies back, lowering their Initiative}\\
Wind blows from the mage, pushing the target back and distracting them.
\iftoggle{verbose}{
	The larger a target is, the harder it is to affect them.
}{}%
\label{windblast}
The spell's total power is equal to its level plus the caster's Intelligence Bonus, minus the target's Strength.
Each point pushes the target back by 1 square and subtracts 1 from their Initiative.
\iftoggle{verbose}{%
	For example, a mage casts a \textit{Wide Wind Blast} spell at two goblins (with Strength -1) and an ogre (with Strength +4).
	Since it's a \textit{Wide} spell, it's cast at level 2, and the +1 Intelligence Bonus makes the total spell-potency 3.%
	\footnote{See page \pageref{wide_enhancement} for casting big spells.}
	The two goblins are pushed back 4 squares and lose 4 Initiative.%
	\footnote{($3 - -1 = 4$)}
	The ogre, however, ignores the spell entirely.
}{}

Anyone pushed back by the spell cannot take Quick Actions again, except Evasion, until their Initiative comes up; so no moving, or speaking is possible until they regain their balance.

\spelllevel

The mage begins to commune with the weather systems and influence how they go. They can even summon localised weather systems from the palm of a hand; mist, sunlight, wind and more are all possible.

\spell{Air Bubble}{Continuous}{Survival}{Ward off missiles or travel underwater in a protective bubble}\\
Weather-workers can summon an air bubble anywhere within range, with a diameter equal to \arabic{spelllevel} squares plus the caster's Wits Bonus. The air bubble can be used to walk underwater without getting wet (though drips through the bubble are common). It will remain despite any damage to its outer `wall' -- penetrating objects simply slip in and out seamlessly. All air bubbles must be summoned while on the land, taking it down below -- any bubbles which begin underwater will simply summon a bubble of stagnant water and will collapse under their own weight once brought onto the land. Air bubbles can also help stop invading winds, mists and such, but with such a limited range their usefulness is also limited.

Any projectiles targeted at the airbubble lose a lot of their power -- arrows, and fireballs both become a little impotent when faced with it.
It provides a total \gls{dr} of \arabic{spelllevel} + Intelligence against all ranged attacks.

\spelllevel

\spell{Forest's Call}{Continuous}{Beast Ken}{Mark someone for a monster encounter}\\
\label{forestsCall}%
The caster makes a call to the forest to come and attack the nearby target.  If the target is a player, the \gls{gm} rolls \arabic{spelllevel} times plus the caster's Intelligence on the local encounter table, and the \gls{pc} faces all encounters within the next day, and typically within the next scene.  The \gls{gm} is encouraged to combine all encounters into one.

If the target is an \gls{npc}, they lose \arabic{spelllevel} \gls{fp} + the caster's Intelligence.
If this leaves the target on 0 \gls{fp}, the target meets with an unfortunate accident next time they enter a natural environment, and dies.

The curse only lasts while it's maintained, and only takes effect in a natural environment where creatures roam -- not in towns or otherworldly environments.

\spell{Telos}{Instant}{Survival}{Make a plant grow to its adult form quickly}\\
The spell reaches out to any plant, dead or alive, and fast-travels it to its natural conclusion.
Seeds grow into plants and blossom, plants grow tall, and older plants whither and die.

The result depends upon the margin.

Poorly made weapons with wooden parts collapse once aged a decade.
Most will collapse after 5.

Bushes targeted by the spell can grow tall instantly, while trees can take decades, or even a century to grow to full height.

\begin{wrapfigure}{r}{.25\textwidth}

	\begin{rollchart}

		Margin & Ageing \\\hline
	
		0 & 1 Year \\
	
		1 & 5 Years \\
	
		2 & 1 Decade \\
	
		3 & 5 Decades \\
	
		4 & 1 Century \\
	
		5 & 2 Centuries \\

	\end{rollchart}

\end{wrapfigure}

The spell must target a complete `thing', and never a piece of a thing.
A basic spell can target a sword, therefore destroying its handle with age, but could not target a door in a house -- the entire house would have to be targeted, or the spell would not work.
Spells massive enough to target a building might affect the exterior, but would to nothing to the interior unless it could target every room within as each room counts as its own area.

\end{multicols}

\sphere{Conjuration}

\begin{multicols}{2}

\noindent
Conjuration deals with changing matter.
It starts by shifting water into mud, or stone into ice.  Later the caster can change types of matter -- liquids into solid, solid metal into air, anything simple.
Higher levels are less limited, and complex items like a bow and arrow, or cart, can be made in an instant, as matter's shape can be changed.  Casters soon also learn how to change matter's location, teleporting items from one place to another.

Conjuration uses various skills to cast, but most commonly Crafts for summoning or changing items, or Survival for water or other simple substances.

Conjuration divides the world into three essential forms -- solid, liquid and gas.  Gases are easiest to work with, liquids come shortly after, and solid objects are usually the most difficult to work with.

Conjuration spells targeting larger items are always more difficult.  The \gls{tn} for anything besides a gas, such as air, is always increased by the \gls{weightrating}.
In the case of living targets, the \gls{weightrating} is always equal to their \gls{hp}, so targeting someone with 6 \gls{hp} would increase the \gls{tn} by 6.

Any character can decide that a conjuration spell targeting them fails by spending 5 \gls{fp}, if that spell would be fatal.

\spelllevel

\spell{Transmutation}{Continuous}{Varies}{Turn solid objects, liquid, or gasses into something simple, of the same type}\\
The mage can turn any single, cohesive, target into another of the same type.  Mist can turn to air, or air can turn into mist.
Ice can turn into rock, and water can turn into sludge.

Food substances, gold coins with complex engravings, bows, and other crafted items are too complicated for this spell -- it only transforms matter into something simple of the same type.

The \gls{tn} is typically 7 plus the target's \gls{weightrating}.

What follows are a number of spells derived from Transmutation.

\spell{Choking Fog}{Continuous}{Survival}{Create noxious gas which causes Fatigue}\\
The caster changes the nearby air into a caustic mess.
When cast outdoors the mist dissipates at the end of the \gls{round}.
When cast in a windy area, the fog disappears instantly.

Anyone Keeping Edgy can hold their breath.
Others gain \arabic{spelllevel} + the cater's Intelligence in Fatigue points each \gls{round}.

The spell affects a single square by default.

\spell{Purify Air}{Continuous}{Survival}{Clear air in a small area}\\
\iftoggle{verbose}{
	Smoke, fog, or any other substance can be purified.
	The spell affects a single square by default.
	Casting this as a \textit{Wide} spell allows a larger area to be cleared.
}{
	Smoke, fog, or any other air impurities can be cleared in a single square.
}

\spell{Stonespell}{Continuous}{Crafts}{Turn any person or other matter to stone, with a TN equal to 7 plus the target's \glsentrytext{weightrating}}\\
The caster changes any solid target to stone, ice, or any other simple, solid, substance.  The \gls{tn} is 7 plus the target's \gls{weightrating}.\footnote{A living target's \gls{weightrating} is equal to their \glsentrytext{hp}.}
Once the spell is over, the target turns back to normal.

Anyone may spend 5 \gls{fp} in order to stipulate that the spell fails.

Fast moving items, such as a spear used in combat, are additionally difficult to target.  When in use, whatever skills the wielder is using add to the \glsentrylong{tn}.

Metal cannot be targeted by this spell.

\spell{Slime}{Continuous}{Survival}{Make any liquid into a slippery slime}\\
The caster turns any nearby liquid into a slippery slime.
Anyone running full speed across the area makes a Dexterity + Athletics roll, \gls{tn} 7 + the caster's Intelligence + Survival.
Anyone simply running (but not at full speed) gains a +2 bonus.
Those who fail, fall over, becoming \textit{prone}.%
\footnote{See page \pageref{prone} for falling prone.}

Some kind of liquid must be in the right place for the spell to work.
Casters acting quickly often carry their own water.  Throwing water requires 8 initiative for using an item, as usual.

\spell{Web}{Continuous}{Survival}{Turn a liquid into a sticky substance - targets roll to be free with Strength + Athletics vs the caster's Intelligence + Survival}\\
The caster turns any liquid into a vicious, sticky substance.
Anyone coming into the liquid gets stuck, and needs to take a full movement action to try to get free.

Casters roll their Intelligence + Survival at a \gls{tn} of 7 + the target's Strength + Athletics. Alternatively, players can avoid being stuck in the web by rolling Strength + Athletics, at \gls{tn} 7 + the caster's Intelligence + Survival.

Anyone can attempt to break free instead of their usual movement action.

Webbing cannot be used instead of rope -- it's too elastic, and tends to snap when stretched.

\enhancement{1}{Meticulous}{Make detailed creations}
The caster can now transform targets into detailed forms.  Air can become a complex, and rich scent.  Solid wood can turn into a sword, or rope.  Water can become beer, wine or even acid.

\enhancement{1}{Metallic}{Target and make metals}
The caster can now target and create basic metals such as copper, bronze, or iron.  Gold and silver cannot be targeted or created, nor can alloys, or weapons adorned with precious metals.

\enhancement{1}{Transient}{Transform any type of matter to any other}
The conjuration spells can now move from any type of matter to any other.  Webbing, slime, or acid can be created from any substance except metals.

Casters turning air into rocks can rain a heavy load down on an enemy, inflicting $1D6$ Damage, plus their Intelligence, plus the spell level.

Living creatures turned using Stone Spell into a solid substance, and then turned into air or water using Form Breach, are dead.

\spelllevel

Level 2 conjuration can use all the enhancements of the previous level.

\spell{Acid}{Continuous}{Academics}{Create caustic acid to burn unclothed targets}\\
The caster can turn any liquid into a potent acid.
If the acid achieves a Vitals Shot (or the target is not wearing clothes), it gets around armour or clothing, and deals $1D6$ Damage plus the spell's level, plus the caster's Intelligence Bonus.
Thin clothing may only provide partial \gls{dr}.

The spell can either be cast a \textit{Transient} spell, in which case it can attack targets by turning air into acid, or it can be cast against a target already covered in some liquid.

Alternatively, if the acid is held in a tough bowl, made or metal or dense wood, it can be thrown like a normal projectile, so long as the range is short (-2 penalty per square's distance).

However the caster creates the acid, targets can dodge like any other missile weapon.

\spell{Prison}{Continuous}{Crafts}{Ice forms around the target, trapping them, unless they move with a Strength + Athletics vs Intelligence + Crafts roll}\\
This spell is simply an example of stacking spell enhancements together.  The caster freezes water around a target, or turns surrounding air to stone, imprisoning them.
While the spell is being cast, the target can attempt to break free as a Quick Action, costing 2 Initiative, by rolling Strength + Athletics.
The \gls{tn} is 7 plus caster's Intelligence Bonus plus Crafts.

If the spell completes, the \gls{tn} to break free increases by 2.

\spelllevel

\spell{Teleport}{Instant}{Academics}{The mage teleports 3 squares + Wits away}\\
The mage teleports the target a short distance -- up to \arabic{spelllevel} squares plus the caster's Wits.  As with many other instant skill spells, the target can cancel the spell by spending 5 \gls{fp}.

If cast as a \textit{Massive} spell, the portal gains can travel across multiple areas, but always remains as around the size of a doorway.

\enhancement{1}{Gated}{Open a portal instead of teleporting}
The mage can not simply teleport something but open a doorway from one place to another, within the normal range.
The magical portals always maintain the same direction, so one which opens facing upwards will always match another which opens facing upwards.

If placed on a surface, it opens seamlessly, as if it were a normal opening.
People can wander into another land entirely and never know it.

The portals are always seamless -- the edges contain no flickering or wobbles.
Portals must always rest upon unchanging surfaces; any movement destroys the circle instantly.

\end{multicols}

\sphere{Enchantment}

\begin{multicols}{2}

\noindent
Enchanters open, tinker with and enslave people's minds. At low levels they learn to charm people, or even let others charm people. Better enchanters can also confuse people to the point of being useless in battle, or to make targets sleep. Finally, the enchanter learns to bend people's will to the point where they are completely subservient to them.

This sphere of magic only works on people with an Intelligence Attribute and works best on humanoids. Casters attempting to affect the strange minds of outsider entities from other planes, the undead or other weird lifeforms should be given an appropriate penalty. Undead are particularly difficult to contact through this spell, especially those who were never human; the \gls{tn} for such a feat should raise by at least +6.

\spelllevel

\spell{Calm}{Continuous}{Empathy}{Remove fear from a target}\\
Enchanters can calm down scared people including those who have failed a Morale Check.
While under the care of an enchanter, all Morale Checks gain a bonus equal to the spell's level plus the Enchanter's Intelligence Bonus.

\spell{Dream Walk}{Continuous}{Empathy}{See a target's Dream}\\
The mage focusses on a dreaming target and perceives their dreams while interacting with them.

Those inside a dream can use any spell, as long as their relevant Skill is equal to the level of sphere they want to employ, as if they were on the Path of Blood.
\iftoggle{verbose}{%
	For example, someone with Survival 1 can use Plantform from the Aldaron sphere (which is a level 1 spell, and uses the Survival Skill).
	Someone with Empathy 2 could use the Enchantment spells \textit{Calm}, and \textit{Focus}, but not \textit{Sleep} (as this is a level 3 spell.)

	Spells which have variable Skills, such as \textit{illusion}, are generally available.
	Someone with Beast Ken would be able to cast illusions of animals, and someone with Crafts would be able to make an illusion of a chest.

	Everyone's total \glspl{mp} determine their Metamagic ability, as usual.
}{}
All dreamers can use their standard spheres in addition to any gained through these lucid dreaming abilities.

The caster can interact normally with the target, and those on good terms can communicate with each other.

Anyone damaged in a dream loses \glspl{mp} instead of \glspl{hp}.
Everyone has a natural \gls{dr} equal to double their Charisma Bonus.
Once they receive damage without having further \glspl{mp} to sacrifice, they wake up.
\iftoggle{verbose}{
	The spell can be used in this way to exhaust people, as it robs them of the ability to recover \glspl{fatigue} while sleeping.
	}{}

A \textit{Wide Dream Walk} spell pulls targets into a single dream space.

\spell{Imbue Soul}{Continuous}{Empathy}{An object gains a small soul, which can be useful for fooling the undead}\\
The caster pours a little life-essence into an object, animal, or anything else.
When used on animals, the creature slowly becomes smarter, though this can take some days to have any real effect.

The spell attracts undead to the target, who feed on the kind of sentient souls that the spell imbues.
Any undead in the area will follow the target, just as if it were a person.
With mindless undead, this works without failure, though intelligent undead can plainly understand that the item is not a person if they can see it properly.

\spell{Fear}{Continuous}{Deceit}{The target suffers a morale penalty of 1 plus caster's Int}\\
\Glspl{npc} hit by this spell suffer a Morale penalty equal to the spell's level plus the caster's Intelligence Bonus.
\Glspl{pc} hit by this spell are not allowed to know their current \gls{fp} total -- the \gls{gm} tracks it instead.

\spell{Reading the Ripples}{Instant}{Vigilance}{Find out the target's Mind Attributes and Code}\\
The enchanter can read any target's Mind Attributes, see which Code or God they follow (if any) and sees all of their Knacks.\footnote{See page \pageref{gods_codes}.}
This will not grant any information about what the target is thinking, merely how capable that mind is and its priorities.

Unwilling targets resist this spell with their Wits + Deceit.

\spell{Sending}{Continuous}{Performance}{Send a psychic message to someone}\\
The enchanter telepathically sends a short message to the target within normal range.
If cast as a \gls{standingspell}, the caster can telepathically send messages for as long as they are within range of the target.

If the enchanter does not have any languages in common with the target then the \gls{tn} is 9 rather than 7.
This communication is one-way only.

\spell{Twitch}{Continuous}{Performance}{Gain Init bonus to spell casting equal to 1 plus Int}\\
The spellcaster focusses on their own mental acuity, gaining a bonus to Initiative for all spell casting.
The bonus is equal to the spell's level plus the caster's Intelligence Bonus.

\spelllevel

\spell{Confusion}{Continuous}{Deceit}{Remove a target's actions for the round, then give an Initiative penalty of the caster's Wits + 2.}\\
The enchanter gives someone a particularly off-putting look and they immediately stops what they were doing and loses their train of thought.
They have trouble articulating exactly what's wrong, but will remain confused for as long as the spell continues.
The spell is sometimes initiated by eye contact, sometimes by song -- any number of social interactions can suffice for transferring the spell's effects.

\iftoggle{verbose}{
	\pic{Roch_Hercka/elvish_enchanter}{\label{roch:enchanter}}
}{}

A resisted roll is made -- the enchanter uses their Intelligence + Deceit Skill while the target uses Wits + Academics.
If the target loses the roll they immediately loses all remaining actions for the turn but can still defend themself; the target's Initiative score instantly reduces to 0.

Each subsequent turn the target makes a resisted roll of Wits + Academics against the mage's Intelligence + Deceit.
Failure indicates that they suffer an Initiative penalty equal to the spell's level plus the caster's Intelligence Bonus.

While the spell is in effect, the target suffers a penalty to all Mental Attributes equal to \arabic{spelllevel} plus the enchanter's Intelligence Bonus; so a mage with Intelligence +3 would inflict a -5 penalty. If the target attempted to cast spells, any rolls would suffer a -5 penalty and any spell-effects which relied on the Intelligence Attribute would suffer as well.

At the end of the scene, targets make one final resisted roll against the enchanter's Intelligence + Deceit (even if the enchanter is no longer present). Failure indicates that the target has forgotten the encounter entirely, including some moments before when the spell began.

\iftoggle{verbose}{
If an \gls{npc} enchanter intends to cast this on a \gls{pc} during a scene, the \gls{gm} is encouraged to simply make the resisted roll for the spell.
If the player fails the roll then the \gls{gm} can infer what probably would have happened had the scene played out and skip to the next scene, telling the player that something important might have happened, but that they cannot remember any of it.

When this spell hits someone out of combat, perhaps during a conversation, targets tend to flap their mouths open and shut like a confused fish as they try to recapture their train of thought. The use of magic will is not obvious to those unfamiliar with such abilities.

}{}

\spell{Focus}{Continuous}{Empathy}{Force a target to repeat whatever they're doing}\\
The target holds the last action performed and repeats it, again and again.
If they were attacking, they will continue attacking until there are no targets left, and then go and look for more.
If the target was attempting to mount a horse but the horse flees, they will chase it until they can no longer move.

The enchanter engages in a resisted roll of their Intelligence + Empathy versus the target's Wits + Academics.
Targets can stop once their original action has become obviously impossible or is unmistakably complete.

\spell{Oath}{Continuous}{Academics}{Force a target to fulfill any promise}\\
The target repeats and emphasises an oath while the caster completes the spell.
For as long as the spell endures, the target cannot break their oath.

A Quick spell allows casters to accept any statement made on the same round.
Even short sentences, such as `I'll find out', or `I'm going to leave at sunrise', can be interpreted as oaths, although if someone does not state \textit{when} they do something, the expected time defaults to any time up until the end of the scene.

\spelllevel

\spell{Sleep}{Continuous}{Empathy}{Make a target instantly sleep. Intelligence + Empathy vs Wits + Academics}\\
Enchanters who want their target to fall asleep can make a resisted Intelligence + Empathy roll against the target's Wits + Academics.
The target can spend 5 \gls{fp} to ignore the results of the spell. A successful spell means that the target has fallen asleep.

\spell{Expectations}{Continuous}{Varies}{The target sees whatever they expect, even if what they expect is wrong}\\
The caster can make someone believe something they were already expecting to see.
If they thought they had beer in their cup, they will continue to drink it, even when it's been replaced by something else.
If they expected to see a dragon in a cavern, they will walk round a corner and believe they are face to face with a dragon.

The caster might look deeply into the target's eyes and force them to hear music which is not in fact there but persists despite all attempt to stop it. They might sing to all present about a dragon, and one particular listener will actually see, feel and smell that dragon.

In all cases a successful illusion will be complete, and the target will make every provision to interact realistically with the imaginary thing, be it a creature, an object or weather condition. It could even be something stranger, such as a box containing a spider's voice, or a statue of a sunrise which glows in unknown colours.

The caster and target make a resisted roll: the caster uses their Intelligence + some Skill relevant to the illusion being created. A caster making a dragon might use Ether Lore, while making an illusory cow would require Beast Ken. The target resists with their Wits and the same Skill as the caster.

The \gls{gm} should make this roll for players, in secret. The target gains a bonus to resist (or the caster takes a penalty) if the illusion is particularly unbelievable (such as a bizarre object or an unexplained dragon). Targets also gain a penalty to resist if they suspect that magic is being used to trick them, which often becomes obvious if lots of people around are insisting that rats are not in fact biting off their toes.

Such mental illusions can inflict Fatigue Points instead of damage, as people's mind creates the damage they expect.
The maximum number of Fatigue points inflicted is equal to the spell's level plus the caster's Intelligence Bonus and multiple castings allow the Fatigue Points to stack up.
These Fatigue Points are healed as normal.
The player may be told that this is Damage, but the \gls{gm} should keep track of it separately to ensure that all the Damage is properly converted once the spell ends.

\spelllevel

\spell{Domination}{Continuous}{Deceit}{The target obeys all commands}\\
The target is given a simple command by the enchanter, consisting of no more words than \arabic{spelllevel} plus the enchanter's Intelligence + Deceit. If the target fails the resisted task of their Wits + Academics against the enchanter's Intelligence + Deceit then they must immediately obey any commands the enchanter gives them.

If the enchanter maintains the spell then the target can reroll at the beginning of each scene to break the spell again, otherwise it ends when the enchanter drops the spell.

\end{multicols}

	\begin{tcolorbox}[arc=1mm,tabularx={llp{.5\textwidth}}]
		Task Bonus & \gls{tn} & \\\hline

		Humiliation & +2 & Any action which would humiliate the target grants a +2 bonus to resist. \\

		Betrayal & +4 & Targets who would otherwise be weak-willed and at the mercy of the enchanter gain a +4 bonus to resist attacking their allies. This bonus can increase up to +6 to resist attacking loved ones such as family and close friends.\\

		Code Violation & Variable & Targets forced to act against their own code or god gain an additional bonus to act equal to the amount of \gls{xp} they would receive for completing the action.
	For example, those following the code of passion would gain 1\gls{xp} for trying a new type of food or drink, so they gain a +1 bonus to resist commands which inhibit their ability to act in this way.
	Those following \gls{wargod} gain 10 \glspl{xp} for bringing down a sufficiently large monster, so they would gain a +10 bonus to resist any enchantment which prohibits them from slaying such quarry.
	This can also be used against the target, with the enchanter gaining a bonus to affect someone with an order if it adheres to the target's code.
\

	\end{tcolorbox}

\begin{multicols}{2}

\noindent
Giving a command can take some time, so in combat, Enchanters have to spend the usual 2 Initiative to speak in order to actually make a target do something, once the spell has been cast.

Some commands are easier to resist than others. Particularly repugnant commands allow the target to reroll to break the spell with a bonus.

\spelllevel

\spell{Mental Bondage}{Continuous}{Deceit}{The target becomes obsessed with the enchanter}\\
The enchanter locks down the target's every thought and turns everything they know to a desire to serve only the enchanter. They will follow any command to the best of their abilities, and if asked why will proclaim an unconditional love for or obedience to the caster.

The target makes a resisted task of their Wits + Academics against the enchanter's Intelligence + Deceit.
Success (from the target's point of view) means that the target breaks the spell but failure (a successful roll on the part of the enchanter) means that the spell is fixed -- for as long as the caster wishes the target will serve them loyally.
Immediate threats to the target's life, such as being told to jump off a cliff or being told to drink something by an enchanter who was previously trying to kill the target call for a reroll, but there is no automatic reroll at the beginning of each scene.
This spell is subject to the same modifiers as the previous level.

Enchanters might use this to turn attacking ogres into a loyal group of warriors to use against other enemies, or simply to turn a favoured artist into a persistent plaything of the local court. This spell may be expensive in terms of \gls{mp} but over time the target may come to loyally serve the enchanter naturally, assimilating the spell into normal, everyday habits. Every month of service prompts a new roll -- success means that nothing happens while if the target fails they must serve the enchanter even after the spell has been cancelled, with full normal effects. Enchanters do not know when their spells have turned into long-term spells, but they can often guess by looking at just when the target has stopped trying to fight the spell.

If the enchanter ever dies, the target can reroll each scene to break the spell.

\spell{Tabula Rasa}{Continuous}{Deceit}{The target forgets everything}\\
The target's memories can be filched -- either selectively or not. The caster specifies (through song, words, or a simple glance) which memories are to be removed. If a target loses access to a Skill due to this spell, they can no longer use it until the spell ends.

The caster uses their Intelligence + Deceit while the target resists with their Wits + Academics.
Success means that the caster has free reign, not to rifle through the target's exact memories, but to specify that anything they wish is lost, up to and including all memories.
The target always retains their first language.

\end{multicols}

\sphere{Fate}

\begin{multicols}{2}

\noindent
Fate deals with divine blessings and luck.
It adds and subtracts luck, shows what the future may hold, and grants \textit{deus ex machine}-style aid.

Bards picture this sphere as a kind of deep intuition, while priests view it as the ability to make requests from the gods.

\spelllevel

\spell{Curse}{Continuous \& Instant}{Deceit}{The target loses $1D6$ + Int \glsentrytext{fp}}\\
The priests calls for the target's death, and then hopes for the world to provide.
The target loses $1D6$ \gls{fp} plus the caster's Intelligence Bonus.
If the target has no \gls{fp} then this spell has no effect.
The mage is allowed to know how many \gls{fp} the target has lost.
The target cannot dodge in any way -- the caster simply rolls their Intelligence + Deceit against \gls{tn} 7.

The target's maximum \gls{fp} are reduced by the spell's level plus the mage's Intelligence Bonus for as long as the spell endures.

\spell{Eyes of Fate}{Continuous}{Empathy}{Read another's current \glsentrytext{fp}}\\
The priest locks into another's fate to see whom the gods deem worthy of special attention, and just how much attention they are getting at the current moment.
Once the spell is cast, the priest knows the current \gls{fp} of the target.

When cast on oneself, this spell grants total immunity to the Enchantment spell, \textit{Fear}.

\spell{Intuition}{Instant}{Varies}{Find out the \glsentrytext{tn} for an action}\\
\iftoggle{verbose}{%
	Players often don't know the \gls{tn} for an action before they try it, but with this spell, the priest may demand to know the \gls{tn}.
	This can be used to figure out how difficult it would be to strike an enemy, or how challenging it would be to spread a rumour around town.
}{%
	The player can ask the \gls{gm} the \gls{tn} for any currently possible action.
}
The Skill used is the same as that being used in the task, so asking about a roll for Crafts means using the Crafts Skill for the spell.

\spell{Lending Hand}{Continuous}{Empathy}{Bless a target with +1 to any skill so long as you have a higher Skill level than the target}\\
The priest blesses a target with +1 to any Skill, so long as the priest has a higher level in that Skill than the target.

\spelllevel

\spell{Auguary}{Instant}{Tactics}{The \glsentrytext{gm} tells you about an upcoming encounter}\\
The character requests guidance about the future and receives a cryptic message from their deity, from dreams, or simply the shape of nearby clouds.

The \gls{gm} should roll for the player so the player is unsure how accurate the information is.

The \gls{gm} might create some riddle, or describe a prophetic vision.
Alternatively, if the Encounters or Side Quests systems are being used, the \gls{gm} may choose to describe an upcoming encounter or read out upcoming boxtext.\iftoggle{verbose}{\footnote{See page \pageref{encounters}.}}{}
If it succeeds, boxtext or encounters can be taken from a different area, or a later encounter.
And if the roll succeeds with a Margin of 4 or more, the player can elect a specific area to receive the boxtext from.
If the roll fails, the \gls{gm} can create misleading information.

If the party radically change their plans in order to avoid an encounter they think sounds bad, the Side Quests should be randomized, leaving some chance they will encounter the same place again.

Characters who continue to cast Auguary receive the same answer each time until they have run into the encounter, or somehow bypassed it.

Nobody with this power ever says ``you cannot change your fate''.  Changing your fate is the entire point of this spell.  Besides, if the spell ever appears to go wrong, the local priests will explain that it actually predicted events correctly.  It was simply your knowledge of the spell that -- somehow or other -- altered what would otherwise have been a fine prediction.

\spell{Blessing}{Instant}{Empathy}{Target regains $1D6 + Int$ \glspl{fp}}\\
The priest blesses the target with the favour of the gods. The target `heals' or regenerates $1D6$ \gls{fp} plus the priest's Intelligence Bonus. This cannot take the target above their maximum \gls{fp} score.

\enhancement{1}{Generous}{Heal a target for additional \glspl{fp}}

The priest heals the target for an additional 2 \glspl{fp}.
These \glspl{fp} stack just like Damage, so $1D6+4$ \glspl{fp} becomes $2D6$ \glspl{fp}.

\spelllevel

\spell{Fortune}{Continuous}{Empathy}{Add +1 to any Skill}\\
The priest blesses a target, who then receives a +1 to any Skill.
This does not stack with any other Fate spells.
This spell can take a character beyond the standard Skill levels.

\spell{Prayer of Gratitude}{Instant}{Academics}{Retrieve 1 \gls{storypoint} after you spend 2 or more}\\
The caster rolls during any scene in which someone spends at least 2 \glspl{storypoint}.
With a successful roll, one \gls{storypoint} is returned to the character.

\spell{Snapback}{Instant}{Tactics}{Start a round over again}\\
The caster casts a spell to determine if some plan will work, and subtly alters fate to ensure it gets its best shot.
Once the spell is cast on a person, the caster can decide to rewind this person's round and try the entire round again.
If the target interacts with anything or anyone not covered by the spell then the spell fails.

\iftoggle{verbose}{%
	The only way to use the spell for a fight is to cover all combatants with a \textit{Wide Snapback}.
	The one-person version might be used on a person picking a lock on a door.
}{}

\spelllevel

\spell{God's Chosen}{Continuous}{Academics}{Increase a target's maximum \glspl{fp} by $4+Int$ along with $2D6 + Int$ \glspl{fp}}\\
The target increases their maximum \glspl{fp} by a number equal to the spell's level, plus the caster's Intelligence Bonus.
The character instantly heals a number of \glspl{fp} equal to $2D6$ plus the caster's Intelligence Bonus.
When the spell ends, the maximum FP return to normal.
The spell does not increase the rate at which \glspl{fp} are regenerated.

\spelllevel

\spell{Divine Favour}{Instant}{Academics}{Spend 1 \glsentrytext{storypoint} in return for 5 to spend immediately}\\
The priest spends 1 \gls{storypoint} and gains an addtional 5 \glspl{storypoint} plus their Intelligence Bonus, which must be spent immediately.
This can be used on a summoning miraculous help, such as a crew of soldiers who have a debt to the priest, or a magical ally.%
\footnote{As usual \gls{gm} is free to veto any ideas, but the player is also free to continue pulling new ideas out.}

\spell{Resurrection}{Instant}{Medicine}{Bring the recently deceased back from the dead}\\
The priest summons the soul of a recently deceased person back to their body.
If they are beyond -3 Hit Points, they must roll a Vitality Check again to stay alive, but this time with a +5 bonus.
There is no roll for the caster -- the spell is automatic and the spell is instant, so the effects need not be maintained.
If the spell is made into a \gls{standingspell} then the effects count as being continuously cast.

The spell also heals the target of a number of \gls{hp} equal to half the Margin.
This cannot bring the target above 0 \gls{hp}.
For example, if a \gls{pc} were at -7 \gls{hp} they would normally make a Vitality Check at \gls{tn} 11.
Adding in the Bonus would make the adjusted \gls{tn} 6.
If the Vitality Check were a roll of 11 then the Margin would be 5 and the character would heal 3 \gls{hp}, going up to -4 \glspl{hp}.
This healing should be understood as a retroactive blessing from the gods, indicating that the Damage sustained was not nearly so bad as was once thought.

The spell must be cast within the same scene as the target lost their last \gls{hp}.

If cast on a member of the undead, the target loses $2D6$ \gls{hp} plus the caster's Intelligence Bonus.
No roll is made, and no protection can be given from \glspl{fp} or \glspl{SP}.

\spell{Mana Lake}{Continuous}{Empathy}{Create a font of mana}\\
The priest spends a \gls{storypoint} to sanctify an area, creating a mana lake.
Forever afterwards, the area spills out mana to be absorbed by anyone nearby with empty mana slots.
The caster rolls at \gls{tn} 12.
Each Margin on the roll means one \glsentrylong{mp} is generated each round, so achieving a `14' on the roll would produce 2 \gls{mp} each round.

\end{multicols}

\sphere{Force}

\begin{multicols}{2}

\noindent
The mage can shape pure energy, pushing and pulling at the world with the power of their will alone. They can create magical shields, pick up weapons and grind targets into the ground as if with an invisible, giant, floating hand.

\spelllevel

\spell{Cage}{Continuous}{Combat}{Levitate a target, so they cannot move. \gls{tn} 7 plus the target's \gls{weightrating}}\\
The mage levitates and traps a target, forcing them to remain where they are, or move as the caster desires.
While powerful, the spell is particularly challenging to cast, as it has a \gls{tn} equal to 7 plus the target's \gls{weightrating}.%
\iftoggle{verbose}{%
\footnote{Everyone's \glsentrytext{weightrating} is equal to their maximum \glspl{hp}.}
}{}

Those caught by the spell count as \textit{prone}, leaving them open to Sneak Attacks.%
\iftoggle{verbose}{%
	\footnote{See page \pageref{prone}.}
}{}

The spell has an effective Speed Bonus equal to its level plus the caster's Intelligence Bonus, so casters can move their quarry just as if the spell were running.
As usual, the target cannot be moved outside of the normal spell range.

\spell{Levitation}{Continuous}{Craft}{Levitate anything with effective Strength of $1 + Int$}\\
The mage focuses on lifting something into the air with pure magical energy.
The spell cannot lift moving, wriggling matter, such as live people or animals.
However, mages can lift themselves into the air if they are content to stay extremely still.%
\footnote{Staying still typically makes spell-casting difficult.}

The spell acts as any person would when lifting things, and has an effective Strength, Dexterity and Speed Bonus equal to the spell's level plus the caster's Intelligence Bonus.
The maximum \gls{weightrating} anyone can lift is equal to their Strength Bonus plus 4, therefore, levitating a cart with a \gls{weightrating} of 10 would require a spell with an effective Strength of +6.

If cast as a \textit{Wide} spell on a single target, each additional square affected adds the spell's level to the total the spell can lift.
\iftoggle{verbose}{%
	For example, a \textit{Wide Levitation} cast at level 3 and covering 3 squares, could lift a total \gls{weightrating} of 9 plus the caster's Intelligence Bonus.
}{}

\spell{Lock}{Continuous}{Craft}{Bind a door shut}\\
The mage can erect a magical force field, similar to mage armour, over a doorway to make it more difficult to break through.
The \gls{tn} to break through the door increases by an amount equal to double the level of the Force sphere being employed plus the mage's Intelligence Bonus.
For example, if a door were at \gls{tn} 8 to burst through, a mage with Intelligence +2 could cast the second level of the Force sphere, raising the \gls{tn} to 14.

Mages can also create barriers of pure force to block passageways without a door, just as with mage armour.
The blockade has a number of \glspl{SP} equal to triple the level of Force sphere being employed, plus the mage's Intelligence Bonus and must be battered through with repeated blows to get through the portal.

\spell{Shunt}{Instant}{Combat}{Push someone back, as per \textit{Wind}.}\\
The caster pushes over objects, or pushes back people.
This spell functions exactly like \textit{Wind Blast}, page \pageref{windblast}.

\spell{Slow Fall}{Continuous \& Instant}{Athletics}{Reduce falling damage}\\
When people (or even items) are falling to their doom, force mages can slow the decent, limiting the Damage from such a fall.
The total spell grants a resistance to any Damage incurred through falling equal to 4 points per level of the Force sphere used, plus the mage's Intelligence score.%
\footnote{$(Level \times 4) + Int$}
Therefore, a mage with Intelligence +2 using the third level of the Force sphere would subtract 14 from any Damage incurred through falling.

If cast as a Quick Spell, it can be cast as a Quick Action, outside the usual Initiative order.

\spell{Telekinetic Fist}{Continuous}{Combat}{Improve unarmed combat damage, gaining an effective Strength of $2 + Int$}\\
The mage uses powerful telekinetic blasts to hold and crumple targets in close combat.
Unarmed attacks using Telekinetic fist count as normal Damage instead of inflicting Fatigue Points.
For the purposes of these attacks, the caster counts as having a Strength Bonus equal to the level of the Force sphere being used, plus the caster's Intelligence Bonus.
For example, someone employing the third level of the Force sphere with Intelligence +3 would count as having +6 Strength, and would inflict $2D6+2$ Damage with unarmed attacks.

\spell{Telekinetic Retreat}{Continuous}{Athletics}{Run away fast, with a bonus of $\arabic{spelllevel} + Int$}\\
Mages can add their mental ability to move things to aid their movement.
Any attempts to move, whether fleeing or just flitting around a room, gain a bonus equal to the level of the Force sphere being employed plus their Intelligence Bonus.
The mage can cast the spell on others and it will automatically push them onwards in whichever direction they are running.

\spelllevel

\spell{Clairvoyance}{Continuous}{Vigilance}{Sense the world without sight}\\
The mage can `feel' by delicately touching things with mental movement rather than actually seeing them. They can see in complete darkness whether underwater or on land.

The mage rolls Intelligence and Vigilance at \gls{tn} 6 plus the spell's level.
The spell covers a progressively larger area depending upon the level used.

Mages able to perceive events multiple areas away make for legendary spies, although the power is limited by the fact that while the make can feel events at a distance, they cannot hear voices or read anything.

Any two mages `looking' at the same area can feel each other's presence and instantly understand that someone else is using Clairvoyance.
They can even identify the other mage with a Wits + Empathy roll.

This spell cannot be cast on others -- the target is what is being felt.

\spell{Dancing Swords}{Continuous}{Combat}{Levitate a weapon with effective Physical Attributes equal to 2}\\
The force mage can make a weapon levitate with the power of their mind. It can float nearby to defend them and even float off to stab at enemies who will be hard pushed to counterattack the wielder when they're standing some distance away.

The caster rolls Intelligence + Combat to levitate the weapon at a \gls{tn} equal to 7 plus the weapon's Weight.
The weapon has effective Strength, Dexterity, and Speed Bonuses equal to the level of the spell being employed minus 1, so using the third level of the Force sphere with a longsword would mean the sword could attack as if it were a person who dealt $1D6+3$ damage, with an Initiative Factor of +3.
It would travel 4 squares in a turn using the mage's movement action.

To use the weapon to attack or defend, the mage must focus, so casting spells in the same round would incur the usual panalties.%
\footnote{See page \pageref{combatcasting} for casting spells while fighting.}

The spell's effective Strength Bonus must be sufficient to lift the weapon without encumbrance, so a mage casting the second level of the Force sphere would have an effective Strength Bonus of 1 and could wield a longsword.
To wield an axe a mage would have to use the fourth level of the Force sphere, gaining an effective Strength Bonus of +3.

While the weapon is next to the caster it can defend the caster using its own stats by using an action to Guard.

If someone wants to grab one of the floating weapons they must roll with their Strike Factor just as when making a grab against any character.
A successful grapple means the weapon is too heavy to lift and the spell ends.

\spell{Mage Armour}{Continuous}{Academics}{Create a magical barrier with $6 + Int$ \glsentrylongpl{sp}}\\
The mage casts a shield of crackling energy around the target to protect from all harm, and most often mages target themselves.
The barrier can shatter if attacked but can take a serious beating before breaking.
Each barrier counts as a having a number of \glsentryfullpl{SP}, which are destroyed by Damage like \gls{fp}, but always before \gls{fp} are targeted.
The target gains a number of \glspl{SP} equal to the level of spell used times 3 plus their Intelligence Bonus.

Those protected by the shield cannot attack others as the shield stops all attacks.
However, casters are able to focus enough to use missile weapons and spells by allowing small breaches in the shield's wall.
\footnote{Allowing a target to use a missile weapon requies complete focus, and a Wits + Empathy roll, and can be performed as a Quick action, costing 2 Initiative Points.}

The shields cannot `split' into bubbles.
When cast wide, it can cover a group of people, but the shield will cover all of them or none.

\textit{For example, Annabel the alchemist has the Force sphere at level 3 and Intelligence +2.
She's low on \gls{mp} so she casts it at level 2, gaining 8 \gls{SP}.
On the very next Initiative Count she's hit for 10 Damage and loses all 8 \gls{SP} then 2 FP.}

Multiple castings do not stack -- only the highest casting it used.

The shield can be placed on others if need be, not only the mage, but this will limit their ability to attack.

Armour does not block Damage going onto \gls{SP} -- the character simply subtracts \gls{SP} without any \gls{dr}. The Mage Armour is not affected by a Vitals Shot -- it protects all around, counting as Perfect armour, although not quite continuously enough to keep out water or gasses. Multiples of such spells do not stack -- only the highest is used.

\spelllevel

\spell{Telekinetic Grasp}{Continuous}{Combat}{Wrestle a target down with psychic force}\\
Force mages can wrestle with people from afar using telekinesis. One major advantage with this sort of wrestling is that the mage does not risk being hit back as they can cast the spell from afar. As per the Grappling rules, the mage first makes a roll to capture the target; they roll Intelligence and Combat while the target resists with their current Evasion Factor. Targets can literally feel the force of the mage's mind around them, often described as a hundred tiny, invisible hands or the feeling of an invisible wave. This force can be parried and pushed back like any normal weapon, so targets can use their full Evasion Factor, including bonuses from using a weapon.

If the spell is successful, it inflicts no Damage nor Fatigue Points, but the target counts as carrying an item with a \gls{weightrating} equal to the level of the Force sphere being used.

For example, a mage using Force level 2, with Intelligence +1 and a Combat Skill of +1, could cast Telekinetic Grasp on a gnome.
The gnome adds their Evasion Factor to the basic \gls{tn} of 7 and then the mage resists this with their Intelligence Bonus plus Combat Skill.
If successful, the gnome would count as carrying an item with a \gls{weightrating} of 2.
Assuming this gnome has the usual Strength Bonus of -2, they would then receive a -4 penalty to their effective Speed Bonus.
Their Initiative Score would suffer and they would accrue additional Fatigue Points each time they attempted to run or fight due to the added \gls{weightrating}.

When cast over a full area, all are effected, and movement becomes extremely difficult.



\end{multicols}

\sphere{Illusion}

\begin{multicols}{2}

\iftoggle{verbose}{
	\toppic{Roch_Hercka/illusion_trogdor}{\label{roch:trogdor}}
}{}

\noindent
Illusions create a facsimile of sounds and sights out of pure magic. The thing created might look like a hat, a coin, a rat or even a dragon at higher levels. Illusions also create convincing sound -- loud echoes, the sound of nearby battle, perhaps even imitating an enemy commander's orders in battle. However, illusions are little more than coloured air and noise -- once touched they fade away. They are frightening and if properly used can defeat armies, but are not perfect weapons by any means.

Seeing through an illusion is always an opposed roll -- the victim uses Wits + Vigilance, while the Illusionist uses Intelligence + some appropriate Skill.
If a \gls{pc} could be tricked by an illusion, the \gls{gm} should always roll for the illusionist, without informing the players.
If someone has a reason to suspect that something is an illusion, they should receive a +2 bonus to resist it.
The party also receive a bonus for multiple people who might spot the illusion, as per the standard Vigilance Skill rules.

Illusionists add different Skills to the roll, depending upon what they are making an illusion of.
An illusion of a cart or sword might require the Craft Skill.
An illusion of a monster might use the Beast Ken Skill.
Specialisations in the correct area are, as usual, a requirement if the caster wants to avoid the usual -1 penalty for lacking the appropriate specialisation.

For example, a gnome creates an illusion of a fleeing gnoll with a great bundle of treasure in his hand, hoping the \glspl{pc} will chase after him immediately.
His Intelligence is +2 and his Academics is at +1 though he has no appropriate specialisations, so the players are rolling at \gls{tn} 9.
The \gls{gm} takes the party member with the highest Wits + Vigilance who has a score of +3 in total.
The next highest score in the party is +2 but nobody else has anything to contribute.
The total is +4\footnote{See the rules on teamwork, page \pageref{teamwork}.} so the \gls{gm} rolls for them and obtains a total of 8 -- that's not enough.
As they begin to run, one of the \glspl{pc} remembers they heard about a gnomish illusionist and asks `Are we chasing an illusion?' -- that puts that final score up to 10; the \gls{tn} is reached and the \gls{gm} informs the player that she sees that the gnoll's feet are not always touching the ground properly, so it must be an illusion.

The \gls{gm} should grant bonuses and penalties to illusions depending upon lighting conditions -- illusions inside a shadowy cottage seen from far away should receive an immense bonus, while far-fetched illusions on a sunny day seen up close might receive a penalty.

Illusionists typically create images of things they are familiar with. Unfamiliar objects, such as an illusionist trying to recreate a dragon while never having seen a dragon, suffer a -2 penalty to the roll, at minimum.

While most people are aware that illusion magic exists and so are suspicious of anything outlandish or out of the ordinary, those who have never heard about illusory magics suffer a -2 penalty to disbelieve.

If someone sees an illusion for what it is then the illusion remains, but of course will have less effect. However, while someone fully believes an illusion to be real, they can be psychosomatically damaged by it simply by believing that it's real. All illusions can inflict a total of 1 Fatigue Point per level of the illusion spell plus one per Intelligence Bonus of the caster. For example, a song mage might sing a griffin illusion into existence; all who are fooled by the illusion can be `attacked' by it, receiving up to 4 Fatigue Points. On later \glspl{round} the song causes no more Fatigue Points, even if it keeps playing, but the bard could then create the illusion of a sword using the first level of the illusion sphere. They could use the sword to attack as usual, but not parry blows. While attacking, they could inflict up to 2 Fatigue Points as people believe they have been wounded by the sword, but subsequent attacks would not increase the amount of Damage.

Illusions must be summoned within the normal range of spells, but once summoned they can travel away from the caster without worry -- so long as they are maintained as \glspl{standingspell}, they endure, no matter how far away the caster might be.

Illusions are typically delicate, and even a single \gls{hp}'s damage will dissipate them.

\spelllevel

\spell{Mana Trick}{Continuous}{Deceit}{Make the target seem like it has more or less mana than it does}\\
The mage places a spell on any item or person, so it seems to have more of fewer \glspl{mp} than it really has.
This circumvents spells such as `Detect Mana'.\footnote{See page \pageref{detectmagic}.}

The caster rolls against a \gls{tn} 7, and each margin allows the apparent total \glspl{mp} of the target to increase or decrease by one level.

\spell{Illusion}{Continuous}{Varies}{Make anything look like something else}\\
The illusionist can make anything look like another of roughly the same size.  A fox can look like a dog, a copper coin can look golden, or a gnome can appear like a gnoll.

Illusionists can use this to hide by making themselves look like a bush, or slip unseen into a party by making themselves look like one of the other guests.

Copying a person requires the Empathy skill, while copying furniture would require the Crafts skill.

Anyone touching an illusion finds that it melts in their hands -- a simple handshake can shatter a basic illusion, and handling fake coins quickly dissipates the magic.

Illusionists cover both sound and appearance.  Illusions crafted for sound can change a nearby river to sound like howling wolves, or make someone's voice come out high-pitched.

Illusionists who speak another language can make someone else's speech sound like that language.
If you speak gnomish, your colleagues can be made to sound like they speak gnomish.
Another spell could make a number of gnomes sound like they're speaking in the common tongue.\footnote{Of course at that point, everyone would understand each other, but have a hard time understanding themselves.}

Seeing through an illusion requires a Wits + Vigilance roll, with a \gls{tn} equal to 7 plus the caster's Intelligence and skill (whatever it happens to be).  Alternatively, when a player rolls for an illusion, the \gls{tn} is 7 plus an opponent's Wits + Vigilance.  Having multiple \glspl{tn} can mean some opponents are fooled and some are not.  Anyone specifically looking out for an illusion can gain a +2 Bonus on the roll, or a +4 if they have reason to suspect that the thing in front of them is an illusion.

Illusions require a caster's full focus in order to remain realistic.
A caster who make his friend look like an elf while his friend talks, would have to pay attention to his friend to make sure the facial movements followed along with the real face.

Illusions can only adjust something's size so much.
Something's \gls{weightrating}/ \glspl{hp} can increase or decrease by a number equal to the spell level plus the caster's Intelligence.
A first level Illusion spell cast with Intelligence +1 could make an elf look like a gnoll, but could not make a gnome look like an ogre.
Similarly, a shortsword could be made to look like a simple dagger, but turning a chainmail suit into a small bird would extremely difficult.

The same applies to sounds -- a babbling brooke can be made to sound like a mellow song, but not like the cries of war, unless the illusionist is particularly proficient.

\enhancement{1}{Independent}{Illusions can be complete fabrications}

Illusions can now be cast without any `base' -- they simply appear on their own.
Coins, dogs, dragons, or more, can be fashioned from nothing.

\enhancement{1}{Solid}{The illusions can be felt, and gain +2 \gls{tn} to spot}

Solid illusions are not all that solid, but they can be touched without disipating and hold all manner of nice details, such as \emph{smelling} right, or stopping smoke from blowing through them.  They are also far more realistic, and increase the \gls{tn} to see through the illusion by 2.

These illusions have a Strength score equal to -6, plus the spell's level, plus the caster's Intelligence.

Solid illusions become an extension of the caster, and any caster can cast a spell \textit{through} the illusion, as if the illusion were the caster.  This might be used to cast an Invocation spell through a dragon illusion, or could employ Force to help an illusory creature lift a sword.

Once even a single point of Damage has been dealt to the illusion, it vanishes.

\enhancement{2}{Negative}{The target becomes invisible}

The illusionist finally learns to make less of something, rather than more.  A single person can be silenced, or made invisible (or both).
An empty patch of ground could suddenly appear to break open, showing a great chasm in the ground.

As usual, the illusion is still delicate, and if the person is struck or disturbed in any way, the illusion dissipates.  Combat rolls, for defence or attack, always break such spells unless they are also make \textit{Solid}.

\spell{Light}{Continuous}{Survival}{Create a light to blind enemies}\\
This replicates the Aldaron spell, \textit{Light}, page \pageref{light}.

\end{multicols}

\sphere{Invocation}

\begin{multicols}{2}

\noindent
This is the first choice of spheres for any battle-mage.
It is designed specifically to destroy targets with balls of lightning and fire.
It also has more subtle uses as casters can extinguish flames, plunging people into darkness.

All Invocation spells are rolled as Projectiles, using the mage's Intelligence Bonus and their Projectiles Skill;
casters must have a Projectiles specialisation in Invocation or receive a -1 penalty to all spells.
The basic \gls{tn} is 7 and the difficulty raises by +1 for every 5 squares away the opponent is, just as with normal missile weapons.
As usual, opponents who are keeping edgy (see page \pageref{edgy}) can use their Speed to resist the attack, adding it to the \gls{tn}.
Alternatively, if a player is keeping edgy, it is they who can attempt to dodge the incoming attack, rolling their Speed at \gls{tn} 7 plus the pyromancer's Intelligence and Projectiles Skill.
Shields' Evasion Bonus can add to the roll to resist such spells.

Just like any other long-range spell, Fireballs and other Invocation spells can succeed in Vitals Shot, bypassing armour, if they strike precisely enough (see page \pageref{vitals}).
Blast-radius spells such as a \textit{Wide Fireball} can inflict a Vitals Shot on multiple people.

\spelllevel

\spell{Extinguish}{Instant}{Survival}{Put out any light source}\\
The mage focuses on any source of fire, and extinguishes it.
Larger fires require a \textit{Wide Darkness} spell.

\spell{Fireball}{Instant}{Projectiles}{Burn an enemy for $1D6 + Int$ Damage}\\
The mage throws out a ball of flaming, crackling light which strikes and burns the target. The Damage is $1D6$ plus the caster's Intelligence.

\subsubsection{Spell Enhancements}

\sidepic[33]{Roch_Hercka/conjuration_left}{\label{roch:invocation}}

\enhancement{1}{Raging}{The fireball deals +2 Damage}
The caster increases the spell's level by one and increases the spell's Damage by 2.  A mage with Intelligence +2, casting Fireball at third level would deal $2D6+2$ Damage.

\enhancement{2}{Internal}{The fire ignores all \gls{dr} and \glspl{SP}}
The pyromancer finally learns how to summon fire upon a target without throwing it -- no ball of flame is thrown, fire simply appears, surrounding the target and instantly covers a target anywhere within normal range. It seeps into soft spots and gets into the chinks in armour, bypassing \gls{dr} entirely, including Perfect armour such as \gls{SP} from Mage Armour. The target cannot resist in any way.

If cast with the \textit{Wide} or \textit{Massive} enhancement, the spell targets everyone inside the area.

\end{multicols}

\sphere{Necromancy}

\begin{multicols}{2}

\noindent
Necromancers summon souls from distant, black realms and place them in appropriate bodies -- those of the once living and now dead. The corpses are sometimes filled with their old hosts, locking people into a state of permanent semi-death, or more often with ravenous and malicious spirits from foreign realms. Mages of this sphere begin by imitating the dead, becoming half dead themselves, which allows them to dissuade malicious spirits from attacking.

\spelllevel

\spell{Ghoul Calling}{Instant}{Medicine}{Summon a hungry spirit into a corpse, creating a ghoul. Maximum $2 + Int$ \glspl{hp}}\\
The mage can create their own ghouls from easily accessible realms of malicious spirits.
Small animals such as cats or frogs are easy, while larger creatures such as humans or basilisks are extremely difficult.
The spell is cast on a corpse and the corpse is imbued with one such malicious spirit.
It retains the Strength score (and therefore \gls{hp}) it had in life.
The corpse has Dexterity, Speed and Wits scores of -2 -- it can run, but not terribly quickly.
The creature has neither Intelligence nor Charisma scores. Most will attack all living things on sight.

The mage rolls their Intelligence + Medicine at \gls{tn} 7 to cast the spell.
Any Medicine specialisations dealing with the affected species (e.g. `gnolls', or `humans'), or specialisations concerning death rituals can be used.

Targets can have a maximum of 2 \glspl{hp} plus the mage's Intelligence Bonus, so a mage with Intelligence +1 could only raise targets with up to 3 \glspl{hp} -- perhaps a cat, or very small gnome.

Once the spell has been cast, it need not be maintained -- once a soul has inhabited a body it remains there like the permanent resident of a house.

\enhancement{1}{Enervated}{The target corpse can go up to 3 \glspl{hp} higher}
The mage adds a level to the spell to increase the maximum number of \glspl{hp} by 3.
A mage with Intelligence +2 could raise a solder with 7 \glspl{hp} with an \textit{Enervated Ghoul}.
With the fourth level, the mage could raise a basilisk with 13 \glspl{hp}.

\enhancement{1}{Cunning}{The mage raises an intelligent spirit}
The caster pulls up not a regular ghoulish spirit, but a ghast -- an intelligent and sinister spirit with a mind of its own.
The spirit begins with Intelligence and Wits scores of -2.
Each use of the \textit{Cunning} enhancement can add 4 points plus the caster's Intelligence Bonus.
These points can be spend on any Trait, e.g. Skills, Attributes, or even Spheres.
However, the caster does not have complete control over which type of spirit is summoned.
For every margin on the roll, the caster designates one point.
The remaining points are assigned by the \gls{gm}.

For example, the player rolls to summon a ghast at \gls{tn} 7, with a roll of 10 -- that's a margin of 3.
The player then assigns 2 points to the Combat Skill and increases Dexterity from -2 to -1.
The \gls{gm} then puts the rest into Wits, so that the creature is better able to defend itself against any attempts to control it.
The mage's Intelligence of +3 means the ghast starts with 7 points, so the remaining 4 points raise the creature's Wits score from -2 to +2.
With a natural Aggression score of +2, the Wits + Aggression is 4, so the \gls{tn} to control the creature with \textit{Command the Dead} is 11.

\spell{Command the Dead}{Continuous}{Academics}{Give any order to the dead, as per the Enchantment sphere.  Intelligence + Academics vs Wits + Aggression}\\
The mage can also command any one undead creature to perform any simple action -- a basic phrase without caveats and no more than one verb.
`Dig',\footnote{The undead are the worst workers due to their stupidity, and typically destroy their own hands before they dig very far.
They can be used for anything, but are not necessarily good for much.}
`kill them all' or `wait here' are all appropriate commands.
To execute the spell, the mage rolls with Intelligence and their Academics score at \gls{tn} 7 -- undead creatures resist with their Wits + Aggression.

This spell replicates all five levels of the enchantment sphere with the mage selecting any effect they wish; however, the mage uses Academics instead of any other Skill because the undead may only be `understood' in some technical sense, and not truly empathised with.

\spell{Preservation}{Instant}{Survival}{Slow something's ageing}\\
\iftoggle{verbose}{%
Trainee artists and necromancers have one thing in common -- fruit.
Students of Necromancy often begin their journey by stopping food from degrading.
}{%
This spell gives a sort of `half-life' to rot, such that any foods, corpses, or anything else affected slow their own ageing process incrementally.
They're not sustained in perfect condition forever, but never quite reach an entirely spoiled stage.
}

\spell{Torpor}{Continuous}{Medicine}{Make the target enter a semi-death state, ignoring Fatigue and gaining \gls{dr} 1}\\
The target enters an altered state of semi-death.
They ignore all Fatigue Point penalties (but can still become suddenly unconscious if the Fatigue Point penalty ever reaches -5).
They gain a natural \gls{dr} of 1.
While this spell is active, no undead will be able to feed from them and most will therefore not wish to attack them.
While this spell is active, the target suffers a -2 penalty to all Charisma checks, though this does not affect \glspl{fp}.

This caster rolls Intelligence + Medicine at \gls{tn} 7 to activate this spell. It can never be cast on others. While the spell is in effect they suffer no ill effects from Fatigue Points but cannot heal them. Once the spell is over, the mage often comes crashing down, collapsing from the weight of the awful things they have done to their body while immune to Fatigue. The caster faces a real danger of death if ever they gain enough Fatigue Points to push them over a -5 penalty; they may not gain the penalty but must make a Vitals Check to avoid death and then make another roll each time they gain Fatigue.

\enhancement{1}{Necrotic}{Increase the \gls{dr} to 2, and see as the dead do}

By adding an additional level to the process, the target can gain the special sight of the undead (in addition to their normal vision).
They can now see all living things, even in the darkness.
Addtionally, the Charisma penalty for the spell raises to -4, as they seem permanently distracted and unable to focus upon the same world that everyone else does.

Additionally, the target's \gls{dr} raises to 2 as the target stops feeling pain altogether.  They can even hold their breath for one minute per spell level.

Targets who die while this spell is in effect raise from the dead as an undead creature.\footnote{This spell cannot raise someone as undead if the necromancer's spell level would not normally allow them to raise a creature of that spell level.}

\spelllevel

\spell{Sickness}{Instant}{Medicine}{The target loses $1D6 - 2 + Int$ \glspl{hp}}\\
Even low level necromancers have the terrifying ability to pull someone's soul out with a simple spell.  The spell inflicts $1D6-2$ Damage directly to the target's \gls{hp}.  \Glsentrylongpl{fp} and \glsentrylongpl{SP} can be bypassed entirely.  The caster adds their Intelligence Bonus to the Damage.

\enhancement{1}{Fetid}{Add 1 to the Damage}

By adding additional levels, the caster can add 1 \gls{hp} to the total Damage.

\end{multicols}

\sphere{Polymorph}

\begin{multicols}{2}

\noindent
The Polymorph sphere of magic allows the mage to grasp at different strands in the tree of life, and move themselves or others along different paths.  Nearby forms include other races, such as elves turning into men, and later shapeshifters learn to turn into bears, hawks or other animals.  Larger men find it easier to turn into large animals such as griffins, while smaller, lighter people find it easier to take on the form of birds.  Master shapeshifters learn to go beyond the great tree of life and turn into arbitrary forms of their chosing, including living fire, or a gust of wind.

Throughout all these forms people maintain a universal `face' -- a kind of likeness which they simply cannot get rid of.
Many conjecture that the face is a facet of one's soul showing in the world.
A ginger person transformed into a cat would become a ginger cat.
A skinny person with short hair who transforms into a sheep will become a skinny, short-haired sheep.
Spotting someone who has been transformed requires a Wits + Empathy roll, with a \gls{tn} of 8 plus the level of the Polymorph sphere being employed; e.g. if an elf used the first level to transform into a gnome the \gls{tn} would be 9, but if the elf used the fifth level to transform into a magma elemental, the \gls{tn} would be 14.

Unwilling targets who are to be transformed with Polymorph can spend 5 FP in order to retroactively stipulate that the spell fails.
The undead are completely immune to the Polymorph sphere.

\begin{figure*}[t]
	\begin{tcolorbox}[arc=1mm,tabularx={lccX}]

	\textbf{Animal} & \textbf{Min Str.} & \textbf{Max Str.} & \textbf{Realistic Enhancements (Optional)} \\\hline

	Cow & 0 & +4 & Quadraped \\

	Badger & -4 & -3 & Quadraped \\

\iftoggle{aif}{
	Basilisk & +5 & +8 & DR 4-6, Quadraped \\
}{}

	Bear & +4 & +5 & DR 2, Quadraped \\

	Beaver & -5 & -4 & Quadraped \\

	Bird/ Bat & -5 & -5 & Flight \\

	Cat & -5 & -5 \\

\iftoggle{aif}{
	Chitincrawler & +3 & +5 & DR 4 \\
}{}
	Deer & 0 & +2 \\

	Donkey & 0 & +3 \\

	Frog & -5 & -5 & Amphibious \\

	Goat & -1 & +2 & DR 2, Quadraped \\

\iftoggle{aif}{
	Griffin & -1 & +2 & Flight \\
}{}

	Horse & +1 & +4 & Quadraped \\

	Large Cat & +1 & +3 & Quadraped \\

	Pig & 0 & +3 & DR 2, Quadraped \\

	Rat & -5 & -5 & Quadraped \\

	Wolf & -2 & +1 & Quadraped \\

\end{tcolorbox}
\end{figure*}

As Polymorph changes people's form it also changes Strength and therefore \gls{hp} maximums.
All \gls{hp} lost to Damage remain as lost \gls{hp} after transformation but might not have any effect.
If a player's maximum \gls{hp} is lowered to the point where they are no longer wounded then all wounds simply vanish, though they are still tracked and reappear once the creature has transformed.
If someone's maximum \gls{hp} increases, once again they count as having lost the same number of \gls{hp}, with no \gls{hp} being gained or lost through the transformation process.
All Fatigue stays where it is and no Fatigue Points which previously gave no penalty move to giving the character a penalty.

The new form granted by a Polymorph spell always feels a little strange, so anyone who transforms suffers a -1 penalty to Dexterity until they get used to the new form.\footnote{Any amount of downtime is a reasonable amount of time.}

Nobody is terribly comfortable holding another creature's form.  Like a newborn lamb, such transformations make people clumsy.

\iftoggle{verbose}{

\begin{exampletext}

	Meldon the elf has 5 \glspl{hp}.
	He takes 3 \glspl{hp} Damage and already has 3 Fatigue Points leaving him with a -1 penalty to all actions.
	He then transforms himself into a bird, lowering to 2 \glspl{hp}.
	He now has zero Damage but retains his -1 penalty due to Fatigue.
	After flying away to safety he rests for a while and heals all his Fatigue Points, but when he turns back into an elf all his old wound reappear as his \gls{hp} increases to the point where they can affect him.

\end{exampletext}
}{}

\spelllevel

\noindent
Enhancements from the first level can be applied to all levels of the Polymorph sphere.

\spell{Animal Transformation}{Continuous}{Beast Ken}{Turn any animal into another, \gls{tn} 7 vs target's Str + \gls{dr}}\\
This spell allows the mage to transform one animal into another.
An animal is defined as any living creature without an Intelligence Bonus.
As before, the mage can increase or decrease the target's Strength Bonus by the spell level,
but have to keep within the normal size-boundaries of the animal.
If a boar has Strength +1, turning it into a bear will require an additional 3 points of Strength, because bears have a minimum Strength of +4.
If the caster instead tries to turn a dangerous bear into a housecat, this is a prohibitively difficult task, as house cats have a difference of at least 9 levels of Strength.

The \gls{tn} for such a transformation is 7 plus the target's Strength + \gls{dr}, as tougher creatures are harder to transform.

Such animal transformations are in shape alone, and do not grant any abilities.  Polymorphing into a bird will not let one fly, and taking the shape of a bear will leave a weakened facsimile of the bear's strong teeth and hide.
All transmformed animals lose all \gls{dr}, regardless of their new form.

\enhancement{1}{Bolstered}{Add your Intelligence Bonus to the Polymorph points}

\iftoggle{verbose}{
While basic shapeshifters base their transformative range on the Polymorph spell's level, a \textit{Bolstered} spell allows the caster to use a number of points equal to the spell's level plus their Intelligence Bonus.
If a shapeshifter cast this spell at the second level, with Intelligence +1, they could lower an animal's Strength Bonus by 3, or could turn a human into a gnome, since that requires a Strength adjustment of 3.

Those with the \textit{Realistic} enhancement also gain a number of \textit{Form Points} equal to the spell level plus the caster's Intelligence Bonus, instead of simply gaining points equal to the Spell's level.}{

The caster uses the spell level plus their Intelligence Bonus to determine all facets of the spell's potence, rather than just the spell's level.

}

\enhancement{1}{Empathic}{Lower the \gls{tn} to 7 and remove Dex penalty}

Advanced shapeshifters can extend a little mana into their understanding of alternate forms, and discard the usual \gls{tn} restrictions.  All \glspl{tn} become 7, and the target no longer suffers a Dexterity penalty for transforming.

\enhancement{1}{Realistic}{Add augmentations, such as claws}

The \textit{Realistic} enhancement allows mages to take on creatures' natural abilities with a number of \textit{Form Points} equal to the spell's level.  When transforming a target into an animal, the form of a bird can allow the target to fly, the form of a bear includes teeth, claws and a thick hide.

The Form Points can each be spent on one of the following:

\begin{itemize}

	\item{Claws \& Teeth: +1 Damage}
	\item{Flight: The creature has wings, and can use them properly.}
	\item{Thick Hide: The animal's thick skin grants \gls{dr} 2.}
	\item{Amphibious}
	\item{Quadraped: The creature can fully utilize four limbs to run at double the normal speed when spending a full round moving}
	\item
	Venom: Once the creature gets a successful hit, the target gains 4 \glspl{fatigue}, then 3 on the next scene, then 2, and so on.

\end{itemize}

When the target is to transform into an animal, all unused points are applied to the target's Speed Bonus.  Someone transforming into a bird with 3 Form Points could use one to gain realistic flight, and then +2 Speed.

When transforming into another race, the target merely loses their racial ability, and gains any racial abilities of the target which are concerned with the body.  For example, elves who transform into dwarves lose their immunity from natural cold, but gain the dwarvish ability to consume strong drink.

\spelllevel

\spell{Race Change}{Continuous}{Medicine}{Turn any humanoid into another race, \gls{tn} 10}

\noindent
The basic Polymorph spell allows someone to turn into another race, so long as the racial difference in Strength is not greater than the spell's level.
When cast at first level, gnolls can turn into humans, humans can turn into dwarves, dwarves can turn into elves, and elves can turn into gnomes.

Once the change has applied, the original racial Bonuses are discarded, and the new racial bonuses applied.
Gnomes who turn into elves gain +1 Strength and +1 Speed, and dwarves who turn into gnolls gain +1 Strength, +1 Speed, but -1 Dexterity.

Various enhancements allow the spell to be cast at a higher level, meaning a skilled Polymorphing gnome could eventually learn to turn into a gnoll.

Changing one's own form is \gls{tn} 7, while changing another's is \gls{tn} 10.

Polymorphing into another race does not grant any of its racial abilities.
Changing one's shape to look like an elf will not grant cold-immunity, and Polymorphing into a human will not allow one walk long distances without fatigue.

\enhancement{1}{Trans Species}{Transform humanoids into animals}

The Polymorpher can now cross the species boundary, making themself or another transform entirely into an animal.

Alternatively, the Polymorpher can turn an animal into a person.
This won't yeild any fantastic results, as animals don't suddenly become intelligent once turned into a gnome or dwarf, but it is possible.
Such creatures start with Intelligence -5 and Charisma 0.

\iftoggle{verbose}{
	\pic{Roch_Hercka/polymorph}{\label{roch:polymorph}}
}{}

This spell is cast at \gls{tn} 12, as it either targets an animal, or makes a person into one (the enhancement \textit{Empathy} changes this to \gls{tn} 7).
It uses the Skill associated with the creature the target will become, so turning a wolf into a man uses Medicine, while turning a man into a wolf requires Beast Ken.

\spelllevel

\spell{Freeform}{Continuous}{Ether Lore}{Turn a target into anything}\\
The shapeshifter can throw off the limits of existing and known creatures, and turn into flamming bulls, acidic clouds, or anything else they might imagine.  The basic \gls{tn} for the spell is 14 as the alternative forms are alien even to those who are capable of adoptin them, but as usual the \gls{tn} can be reduced by other enhancements.

As with the \textit{Realistic} enhancement, the caster gains a number of Form Points equal to the spell's level.  The caster can spend \textit{2} form points to purchase any of the following:

\needspace{3em}

\begin{itemize}

	\item{Massive Claws \& Teeth: +2 Damage.}
	\item{Impenetrable Hide: +4 \gls{dr}.}
	\item{Etherial Form: The caster turns into a thick smoke or mist, becoming immune to almost all physical damage.}
	\item{Fiery Form: The caster's body is composed mostly of acid, fire, or some other dangerous substance.
	All grappling attacks deal $1D6$ damage from the hold itself.}
	\item{Many arms: You have many arms, and if they can grasp weapons, then you can have a third and fourth weapon.  The third weapon adds one quarter of its Evasion Bonus.  The weapons otherwise work as per standard \textit{Dual Wielding} rules (see page \pageref{dualWielding}).}

\end{itemize}

\iftoggle{verbose}{
	\pic{Studio_DA/fire_form}{\label{da:fire}}
}{}

\end{multicols}

\stopcontents[magic]


