\chapter*{Introduction}

\begin{multicols}{2}

BIND\footnote{`BIND' stands for `BIND is not D\&D'.} is a zero to hero RPG, with an emphasis on getting an output quickly, and keeping players' decisions in the loop.

Character backstories can be skipped at the start, and thrown in during play, when players know more about the world.

Character creation is random by default, so players have no expectation to understand the entire world before starting play (though you have the option to build a character).

Combat is focussed on giving players real choices, and typically ends quickly as enemies have few \glsentrylong{hp}.

If you're looking for a premade campaign world, find yourself a copy of \textit{Adventures in Fenestra}.

\end{multicols}

\section*{Special Thanks \ldots}

\begin{multicols}{2}

\subsection*{to the Artists}

Neil McDonnell for the basic photograph which became the Polymorph image,

\paragraph{Boris Pecikozi\'c} for Thenton's Story images, (pages 
\pageref{boris:jump}, 
\pageref{boris:brawl}, 
\pageref{boris:meet}), 

%\paragraph{Brian Garabrant} for the cover images

\paragraph{Roch Hercka} for the myriad wonderful pencil sketches (pages 
\pageref{roch:races}, 
\pageref{roch:dwarf}, 
\pageref{roch:stances}, 
\pageref{roch:vitals}, 
\pageref{roch:xp1}, 
\pageref{roch:xp2}, 
\pageref{roch:enchanter}, 
\pageref{roch:polymorph}, 
\pageref{roch:runes}, 
\pageref{roch:light}
).
Find him at artstation.com/hertz.

\subsection*{and to the playtesters} Marri Russell, Ross Oliver, Reiss McGee, David Smith, Michael Dyson, Ryan Trotter and Maggie Anderson; also thanks to Ari-Matti Piippo for his insightful comments.

\end{multicols}

\section*{The Right to Improve}

\begin{multicols}{2}

This book has some serious problems, and that's fine.  I've put this under a share-alike licence,\footnote{\tt GNU General Public License 3 or (at your option) any later version.} so anyone can grab a copy of the basic \LaTeX~ document it's written in and change things.  This isn't the Open Gaming Licence of D20 where they magnanimously allow you to use their word for a mechanic and let you publish things for their products -- this is a publicly owned book.

No longer do imaginative \acrshortpl{gm} have to scribble their inspired house rules onto the back of an old banking statement and cellotape it to the last page of the core book.
Instead, you have the complete source documents, and can modify it as you please, creating a cohesive book.
If you spot an error, you can correct it.
If you want to add a couple of spells, it's no problem.
Just download the source from gitlab.com/bindrpg/, download a Latex editor, and make the changes you want.
Once you're happy with your changes, you might even send it off to a printing shop for a copy of your own version.

And if you happen to make some useful additions, or even deletions, be sure to add them to another git project, where others can benefit from your genius.

With a little work, we could get real community-based RPG.  Something that's always free, something that gets a new edition as and when people want, with just the changes that people want -- a continuously evolving work.

This particular version was last revised on \today.

\end{multicols}

